% !TeX root = FutureWork.tex
\documentclass[../Dissertation.tex]{subfiles}

\begin{document}

\section{Future Work}

\subsection{Strong Dinaturality}
Petri nets, within the developed application, correspond to dinaturality conditions of parametrically polymorphic types. However, this  

Given two functors $F, G : \mathbb{C}^{op} \times \mathbb{C} \rightarrow \mathbb{D}$, the structure-preserving condition of a dinatural transformation $\phi : F \rightarrow G$ is described by the following commutative diagram:
\begin{equation}\label{dinaturality}
  \begin{tikzpicture}[baseline= (a).base]
    \node[scale=1.0] (a) at (0,0) {
      \begin{tikzcd}
        &  F (X, X) \arrow[rr, "\phi_{X}"] \arrow[rd, "{F(id, f)}"]
        &
        &  G (X, X) \arrow[dr, "{G (id, f)}"]
        &
        \\ Y \arrow[ur, "p"] \arrow[dr, swap, "q"]
        &
        & F(X, X')
        &
        &  G (X, X')
        \\
        &  F (X', X') \arrow[ru, "{F(f, id)}"] \arrow[rr, swap, "\phi_{X'}"]
        &
        &  G (X', X') \arrow[ur, swap, "{G (f, id)}"]
        &
      \end{tikzcd}.
    };
  \end{tikzpicture}
\end{equation}

\subsection{Operational Improvement}

\end{document}