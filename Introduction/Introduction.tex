% !TeX root = Introduction.tex
\documentclass[../Dissertation.tex]{subfiles}

\begin{document}

\section{Introduction}
Think of a number, which will hereby be denoted $x$. Does there exist a number $y$, such that $x + y = 0$? Perhaps the most natural response to this question is to confirm that there exists a $y$, given by $y = -x$. However, a precise response to this question is predicated on the answer to a more philosophical question. What is a number?
\par
The answer to the latter question will assuredly determine the former. For example, presume that `number' is to be interpreted as `real number'. It is certainly true that for any real number $x$, there exists another real number $y$, such that $x + y = 0$. However, if `number' were instead to be interpreted as `natural number', no such $y$ exists. Quantifying the domain over which $x$ and $y$ range, is necessary to resolve the ambiguity. 
\par
In the context of programming languages, 

\end{document}