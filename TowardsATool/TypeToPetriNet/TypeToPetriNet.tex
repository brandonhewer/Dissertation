% !TeX root = HindleyMilner.tex
\documentclass[../../Dissertation.tex]{subfiles}

\begin{document}

\subsection{Types to Petri nets}\label{sec:TypeToPetri}
As formerly discussed, the construction of a Petri net from a type requires the additional information of a naturality type. A naturality type, represented by two functions, determines which instances of a type variable are to be equated. Furthermore, any appropriate formulation of a Petri net model, must be capable of describing coherence conditions obtained by composition.

\begin{definition}\label{def:petricomp}
  Let $\tau_1,\ \tau_2,\ \ldots,\ \tau_n$ be types, and $t_1,\ t_2,\ \ldots,\ t_{n-1}$ be natural numbers.  Recall the functor representation of a type given in Corollary \ref{cor:functor}. Define
  \begin{gather*}
    \Theta_i = \mathcal{V}(\tau_i),\quad 
    \lVert \Theta \rVert_m = \sum_{i=1}^{m}|\Theta_i|,\quad
    \lVert t \rVert_m = \sum_{i=1}^{m}t_i.
  \end{gather*} 
  Denote $[x_1,x_2,\ldots,x_n]$ as $[\![x]\!]_n$. Given functions
  \begin{gather*}
    f_1 : [|\Theta_1|] \rightarrow [t_1],\ f_2 : [|\Theta_2|] \rightarrow [t_2],\ \ldots,\ f_{n-1} : [|\Theta_{n-1}|] \rightarrow [t_{n-1}],\\
    g_1 : [|\Theta_2|] \rightarrow [t_1],\ g_2 : [|\Theta_3|] \rightarrow [t_2],\ \ldots,\ g_{n-1} : [|\Theta_{n}|] \rightarrow [t_{n-1}],
  \end{gather*}
  the Petri net $\mathcal{P}([\![\tau]\!]_n,\; [\![f]\!]_{n-1},\; [\![g]\!]_{n-1})$ is defined as follows
  \begin{flalign*}
    |P| = \lVert \Theta \rVert_n,\ |T| &= \lVert t \rVert_{n-1},\\\\
    \forall k \in [n - 1]\ \forall l \in [|\Theta_k|],\ m \in [t_k] :\\
    D^-_{i,j} &=
    \begin{cases}
      1 & \text{if } \Theta_{k,l} = + \land f_k(l) = m\\
      0 & \text{otherwise}
    \end{cases},\\
    D^+_{i,j} &=
    \begin{cases}
      1 & \text{if } \Theta_{k,l} = - \land f_k(l) = m\\
      0 & \text{otherwise}
    \end{cases},\\
    \text{where } i &= l + \lVert \Theta \rVert_{k-1},\ 
    j = m + \lVert t \rVert_{k-1},\\\\
    \forall k \in [n - 1]\ \forall l \in [|\Theta_{k+1}|],\ m \in [t_k] :\\
    D^-_{i,j} &=
    \begin{cases}
      1 & \text{if } \Theta_{k+1,l} = - \land g_k(l) = m\\
      0 & \text{otherwise}
    \end{cases},\\
    D^+_{i,j} &=
    \begin{cases}
      1 & \text{if } \Theta_{k+1,l} = + \land g_k(l) = m\\
      0 & \text{otherwise}
    \end{cases},\\
    \text{where } i &= l + \lVert \Theta \rVert_k,\ 
    j = m + \lVert t \rVert_{k-1},\\\\
    \forall k \in [n]\ \forall l \in [|\Theta_k|] : M^0_i &=
    \begin{cases}
      0 & \text{if } \exists j : D^+_{i,j} = 1\\
      1 & \text{otherwise}
    \end{cases},\\
    \text{where } i &= l + \lVert \Theta \rVert_{k-1}.
  \end{flalign*}
\end{definition}

\begin{remark}
  The complete definition of any well-formed Petri net, corresponding to coherence conditions in System $F_\kappa$, is given in Definition \ref{def:petricomp}. While this definition may at first appear nebulous, use cases and examples will be provided. This general definition will be necessary for formalising the operation of composition.
\end{remark}

\begin{example}
  Consider a type
  \begin{equation*}
    \tau = \forall \alpha.(\alpha \rightarrow \alpha) \rightarrow \alpha.
  \end{equation*}
  Parametricity implies that $\tau$ is a dinatural transformation between the $\rightarrow$ and $Id$ functors. In order to generate a Petri net model corresponding to the naturality condition on $\tau$, two additional functions are required. The first function assigns a number to each type variable in $(\alpha \rightarrow \alpha)$, the second acts identically on $\alpha$. This will frequently be termed a naturality type, and defines which functor arguments are to be equated.
  \par
  The default naturality type is defined by setting both functions to $n \mapsto 1$, thus equating all functor arguments. The Petri net corresponding to the default naturality condition of $\tau$, is formally denoted
  \begin{equation*}
    \mathcal{P}([(\alpha \rightarrow \alpha),\ \alpha],\ [n \mapsto 1],\ [n \mapsto 1]).
  \end{equation*}
  This Petri net may be graphically depicted by
  \begin{center}
    \begin{tikzcd}
    {\tikz[baseline=(char.base)]{\node[shape=circle,draw,minimum size=20pt,inner sep=0pt,fill=lightgray] (char) {};}} &  & {\tikz[baseline=(char.base)]{\node[shape=circle,draw,minimum size=20pt,inner sep=0pt,fill=white] (char) {$f$};}} \arrow[ld] \\
      & {\tikz[baseline=(char.base)]{\node[shape=rectangle,draw,minimum size=12pt,inner sep=0pt,fill=black] (char) {};}} \arrow[lu] \arrow[d] &  \\
      & {\tikz[baseline=(char.base)]{\node[shape=circle,draw,minimum size=20pt,inner sep=0pt,fill=white] (char) {};}} & 
    \end{tikzcd}.
  \end{center}
\end{example}

\begin{definition} The destructuring of a type in System $F_\kappa$, into a list of the constituent type variables is formally defined by
\begin{flalign*}
  \mathcal{Q} (c) &= [\ ]\\
  \mathcal{Q} (\alpha) &= [\alpha]\\
  \mathcal{Q} (\tau_1 \rightarrow \tau_2) &= \mathcal{Q} (\tau_1) \mdoubleplus\mathcal{Q} (\tau_2)\\
  \mathcal{Q} (\tau_1 \times \tau_2) &= \mathcal{Q} (\tau_1) \mdoubleplus \mathcal{Q} (\tau_2)\\
  \mathcal{Q} (T\ \tau_1\ \tau_2\ \ldots\ \tau_n) &= \mathcal{Q} (\tau_1) \mdoubleplus \mathcal{Q} (\tau_2) \mdoubleplus \ldots \mdoubleplus \mathcal{Q} (\tau_n)\\
  \mathcal{Q} (F\ \tau_1\ \tau_2\ \ldots\ \tau_n) &= \mathcal{Q} (\tau_1) \mdoubleplus \mathcal{Q} (\tau_2) \mdoubleplus \ldots \mdoubleplus \mathcal{Q} (\tau_n)
\end{flalign*}
\end{definition}

\begin{example}
  Consider a type
  \begin{equation*}
    \tau = \forall \alpha. \forall \beta.(\alpha \rightarrow \beta) \rightarrow \alpha \rightarrow \beta.
  \end{equation*}
  Destructuring $\tau$ into its constituent type variables yields
  \begin{equation*}
    \mathcal{Q}(\tau) = [\alpha,\ \beta,\ \alpha,\ \beta].
  \end{equation*}
\end{example}

\begin{definition}
  Let $\sigma$ be a type in System $F_\kappa$ and $\alpha$ be a type variable. Define $\theta = \mathcal{V}(\sigma)$. Recall the ramification of substitution on a naturality type, given in Definition \ref{def:subtype}. Given a Petri net model
  \begin{flalign*}
    \mathbf{A} = \mathcal{P}([\![\tau]\!]_n,\; [\![f]\!]_{n-1},\; [\![g]\!]_{n-1}),
  \end{flalign*}
  the operation of substitution on $\mathbf{A}$ is defined as
  \begin{flalign*}
    \mathbf{A}[\sigma/\alpha] &= \mathcal{P}([\![\tau']\!]_n,\; [\![f']\!]_{n-1},\; [\![g']\!]_{n-1})\\
    \text{where }&\\
    \tau'_i &= \tau_i[\sigma/\alpha],\ 
    v_i = |\mathcal{V}(\tau_i)|,\ 
    q_i = \mathcal{Q}(\tau_i),\\
    f'_i &= \bigcup\limits_{k=1}^{v_i} \{(|\theta| * (k - 1) + j,\ f_i(k) + j - 1)\ |\ j \in \theta \land q_{ij} = \alpha \},\\
    g'_{i-1} &= \bigcup\limits_{k=1}^{v_i}\{(|\theta| * (k - 1) + j,\ g_i(k) + j - 1)\ |\ j \in \theta \land q_{ij} = \alpha \}.
  \end{flalign*}
\end{definition}

\begin{example}\label{ex:diag}
  Consider the diagonal function $\forall \alpha.\alpha \rightarrow \alpha \times \alpha$. Given a default naturality type, the corresponding Petri net, defined by
  \begin{equation*}
    \mathbf{D} = \mathcal{P}([\alpha,\ \alpha \times \alpha],\ [n \mapsto 1],\ [n \mapsto 1]),
  \end{equation*}
  may be graphically depicted as
  \begin{center}
    \begin{tikzcd}
    & {\tikz[baseline=(char.base)]{\node[shape=circle,draw,minimum size=20pt,inner sep=0pt,fill=white] (char) {$f$};}} \arrow[d] &  \\
    & {\tikz[baseline=(char.base)]{\node[shape=rectangle,draw,minimum size=12pt,inner sep=0pt,fill=black] (char) {};}} \arrow[ld] \arrow[rd] &  \\
    {\tikz[baseline=(char.base)]{\node[shape=circle,draw,minimum size=20pt,inner sep=0pt,fill=white] (char) {};}} &  & {\tikz[baseline=(char.base)]{\node[shape=circle,draw,minimum size=20pt,inner sep=0pt,fill=white] (char) {};}}
    \end{tikzcd}.
  \end{center}
  Substituting $\alpha \rightarrow \alpha$ for $\alpha$ yields
  \begin{flalign*}
    \mathbf{D}[\alpha \rightarrow \alpha/\alpha] &= \mathcal{P}([\alpha \rightarrow \alpha,\ (\alpha \rightarrow \alpha) \times (\alpha \rightarrow \alpha)],\ [f],\ [g]),\\
    &\ \ \ \ \text{where } f = n \mapsto n\\
    &\ \qquad\quad\ \ \; g = \{(1,1),\ (2,2),\ (3,1),\ (4,2)\}.
  \end{flalign*}
  which may be graphically depicted as
  \begin{center}
      \begin{tikzcd}
        & {\tikz[baseline=(char.base)]{\node[shape=circle,draw,minimum size=20pt,inner sep=0pt,fill=lightgray] (char) {};}} & {\tikz[baseline=(char.base)]{\node[shape=circle,draw,minimum size=20pt,inner sep=0pt,fill=white] (char) {$f$};}} \arrow[d] &  \\
        & {\tikz[baseline=(char.base)]{\node[shape=rectangle,draw,minimum size=12pt,inner sep=0pt,fill=black] (char) {};}} \arrow[u] & {\tikz[baseline=(char.base)]{\node[shape=rectangle,draw,minimum size=12pt,inner sep=0pt,fill=black] (char) {};}} \arrow[ld] \arrow[rd] &  \\
       {\tikz[baseline=(char.base)]{\node[shape=circle,draw,minimum size=20pt,inner sep=0pt,fill=lightgray] (char) {$f$};}} \arrow[ru] & {\tikz[baseline=(char.base)]{\node[shape=circle,draw,minimum size=20pt,inner sep=0pt,fill=white] (char) {};}} & {\tikz[baseline=(char.base)]{\node[shape=circle,draw,minimum size=20pt,inner sep=0pt,fill=lightgray] (char) {$f$};}} \arrow[lu] & {\tikz[baseline=(char.base)]{\node[shape=circle,draw,minimum size=20pt,inner sep=0pt,fill=white] (char) {};}}
       \end{tikzcd}.
  \end{center}
\end{example}

\begin{definition}
  Given two Petri net models 
  \begin{flalign*}
    \mathbf{A} = \mathcal{P}([\![\tau]\!]_n,\; [\![f]\!]_{n-1},\; [\![g]\!]_{n-1}),\\
    \mathbf{B} = \mathcal{P}([\![\sigma]\!]_m,\; [\![h]\!]_{m-1},\; [\![k]\!]_{m-1}),
  \end{flalign*}
  the operation of pasting the domain of $\mathbf{B}$ along the codomain of $\mathbf{A}$, is defined formally by
  \begin{flalign*}
    \mathbf{B} \bullet \mathbf{A} = \mathcal{P}([\![\tau]\!]_{n-1} \mdoubleplus [\![\sigma]\!]_{m},\; [\![f]\!]_{n-1} \mdoubleplus [\![h]\!]_{m-1},\; [\![g]\!]_{n-1} \mdoubleplus [\![k]\!]_{m-1}).
  \end{flalign*}
\end{definition}

\begin{example}
  Observe the Petri net, denoted $\mathbf{D}$, given for the diagonal function, $\forall \alpha.\alpha \rightarrow \alpha \times \alpha$, in Example \ref{ex:diag}. Consider a second type $\forall \alpha.\alpha \times \alpha \rightarrow \alpha$. Given a default naturality type, the corresponding Petri net, defined by
  \begin{equation*}
    \mathbf{C} = \mathcal{P}([\alpha \times \alpha,\ \alpha],\ [n \mapsto 1],\ [n \mapsto 1]),
  \end{equation*}
  may be graphically depicted as
  \begin{center}
    \begin{tikzcd}
    {\tikz[baseline=(char.base)]{\node[shape=circle,draw,minimum size=20pt,inner sep=0pt,fill=white] (char) {$f$};}} \arrow[rd] &  & {\tikz[baseline=(char.base)]{\node[shape=circle,draw,minimum size=20pt,inner sep=0pt,fill=white] (char) {$f$};}} \arrow[ld] \\
      & {\tikz[baseline=(char.base)]{\node[shape=rectangle,draw,minimum size=12pt,inner sep=0pt,fill=black] (char) {};}} \arrow[d] &  \\
      & {\tikz[baseline=(char.base)]{\node[shape=circle,draw,minimum size=20pt,inner sep=0pt,fill=white] (char) {};}} & 
    \end{tikzcd}.
  \end{center}
  Pasting the domain of $\mathbf{C}$ along the codomain of $\mathbf{D}$, is given by
  \begin{equation*}
    \mathbf{C} \bullet \mathbf{D} = \mathcal{P}([\alpha,\ \alpha \times \alpha,\ \alpha],\ [n \mapsto 1,\ n \mapsto 1],\ [n \mapsto 1,\ n \mapsto 1]),
  \end{equation*}
  and may be graphically depicted as
  \begin{center}
    \begin{tikzcd}
      & {\tikz[baseline=(char.base)]{\node[shape=circle,draw,minimum size=20pt,inner sep=0pt,fill=white] (char) {$f$};}} \arrow[d] &  \\
      & {\tikz[baseline=(char.base)]{\node[shape=rectangle,draw,minimum size=12pt,inner sep=0pt,fill=black] (char) {};}} \arrow[ld] \arrow[rd] &  \\
     {\tikz[baseline=(char.base)]{\node[shape=circle,draw,minimum size=20pt,inner sep=0pt,fill=white] (char) {};}} \arrow[rd] &  & {\tikz[baseline=(char.base)]{\node[shape=circle,draw,minimum size=20pt,inner sep=0pt,fill=white] (char) {};}} \arrow[ld] \\
      & {\tikz[baseline=(char.base)]{\node[shape=rectangle,draw,minimum size=12pt,inner sep=0pt,fill=black] (char) {};}} \arrow[d] &  \\
      & {\tikz[baseline=(char.base)]{\node[shape=circle,draw,minimum size=20pt,inner sep=0pt,fill=white] (char) {};}} & 
     \end{tikzcd}.
  \end{center}
\end{example}

\begin{definition}\label{def:petricomposition}
  Given two Petri net models 
  \begin{flalign*}
    \mathbf{A} = \mathcal{P}([\![\tau]\!]_n,\; [\![f]\!]_{n-1},\; [\![g]\!]_{n-1}),\\
    \mathbf{B} = \mathcal{P}([\![\sigma]\!]_m,\; [\![h]\!]_{m-1},\; [\![k]\!]_{m-1}),
  \end{flalign*}
  their composition, denoted $\mathbf{B} \cdot \mathbf{A}$, is given by
  \begin{flalign*}
    \mathbf{B} \cdot \mathbf{A} = B[s] \bullet A[s] \text{ where } s = U(\tau_n,\ \sigma_1).
  \end{flalign*}
\end{definition}

\begin{example}
  Once again observe the Petri net, denoted $\mathbf{D}$, of the diagonal function, provided in Example \ref{ex:diag}. Consider the type 
  \begin{equation*}
    \forall \alpha.(\alpha \rightarrow \alpha) \times (\alpha \rightarrow \alpha) \rightarrow \alpha. 
  \end{equation*}
  Given a default naturality type, the corresponding Petri net, defined by
  \begin{equation*}
    \mathbf{E} = \mathcal{P}([(\alpha \rightarrow \alpha) \times (\alpha \rightarrow \alpha),\ \alpha],\ [n \mapsto 1],\ [n \mapsto 1]),
  \end{equation*}
  may be graphically depicted as
  \begin{center}
    \begin{tikzcd}[column sep=tiny]
      {\tikz[baseline=(char.base)]{\node[shape=circle,draw,minimum size=20pt,inner sep=0pt,fill=lightgray] (char) {};}} &  & {\tikz[baseline=(char.base)]{\node[shape=circle,draw,minimum size=20pt,inner sep=0pt,fill=white] (char) {$f$};}} \arrow[rd] &  & {\tikz[baseline=(char.base)]{\node[shape=circle,draw,minimum size=20pt,inner sep=0pt,fill=lightgray] (char) {};}} &  & {\tikz[baseline=(char.base)]{\node[shape=circle,draw,minimum size=20pt,inner sep=0pt,fill=white] (char) {$f$};}} \arrow[llld] \\
       &  &  & {\tikz[baseline=(char.base)]{\node[shape=rectangle,draw,minimum size=12pt,inner sep=0pt,fill=black] (char) {};}} \arrow[lllu] \arrow[ru] \arrow[d] &  &  &  \\
       &  &  & {\tikz[baseline=(char.base)]{\node[shape=circle,draw,minimum size=20pt,inner sep=0pt,fill=white] (char) {};}} &  &  & 
      \end{tikzcd}.
  \end{center}
  The composition of $\mathbf{E}$ and $\mathbf{D}$ is given by
  \begin{flalign*}
    \mathbf{E} \cdot \mathbf{D} &= \mathcal{P}([\alpha \rightarrow \alpha,\ (\alpha \rightarrow \alpha) \times (\alpha \rightarrow \alpha),\ \alpha],\ [f_1,\ f_2],\ [g_1,\ g_2])\\
    & \quad \ \text{where } f_1 = n \mapsto n\\
    & \qquad \qquad \: f_2 = n \mapsto 1\\
    & \qquad \qquad \: g_1 = \{(1,1),\ (2,2),\ (3,1),\ (4,2)\}\\
    & \qquad \qquad \: g_2 = n \mapsto 1,
  \end{flalign*}
  and may be graphically depicted as
  \begin{center}
    \begin{tikzcd}[column sep=tiny]
      &  & {\tikz[baseline=(char.base)]{\node[shape=circle,draw,minimum size=20pt,inner sep=0pt,fill=lightgray] (char) {};}} &  & {\tikz[baseline=(char.base)]{\node[shape=circle,draw,minimum size=20pt,inner sep=0pt,fill=white] (char) {$f$};}} \arrow[d] &  &  \\
      &  & {\tikz[baseline=(char.base)]{\node[shape=rectangle,draw,minimum size=12pt,inner sep=0pt,fill=black] (char) {};}} \arrow[u] &  & {\tikz[baseline=(char.base)]{\node[shape=rectangle,draw,minimum size=12pt,inner sep=0pt,fill=black] (char) {};}} \arrow[lld] \arrow[rrd] &  &  \\
     {\tikz[baseline=(char.base)]{\node[shape=circle,draw,minimum size=20pt,inner sep=0pt,fill=lightgray] (char) {};}} \arrow[rru] &  & {\tikz[baseline=(char.base)]{\node[shape=circle,draw,minimum size=20pt,inner sep=0pt,fill=white] (char) {};}} \arrow[rd] &  & {\tikz[baseline=(char.base)]{\node[shape=circle,draw,minimum size=20pt,inner sep=0pt,fill=lightgray] (char) {};}} \arrow[llu] &  & {\tikz[baseline=(char.base)]{\node[shape=circle,draw,minimum size=20pt,inner sep=0pt,fill=white] (char) {};}} \arrow[llld] \\
      &  &  & {\tikz[baseline=(char.base)]{\node[shape=rectangle,draw,minimum size=12pt,inner sep=0pt,fill=black] (char) {};}} \arrow[lllu] \arrow[ru] \arrow[d] &  &  &  \\
      &  &  & {\tikz[baseline=(char.base)]{\node[shape=circle,draw,minimum size=20pt,inner sep=0pt,fill=white] (char) {};}} &  &  & 
     \end{tikzcd}.
  \end{center}
\end{example}

\begin{definition}\label{def:instantunit}
  Given a type $\tau = \forall \alpha_1.\forall \alpha_2\ldots\forall \alpha_n.\sigma$, define
  \begin{equation*}
    \tau\{\alpha_i\} = \forall \alpha_i.\sigma[1/\alpha_1]\ldots[1/\alpha_{i-1}][1/\alpha_{i+1}]\ldots[1/\alpha_n],
  \end{equation*}
  i.e., the type generated by instantiating all type variables in $\tau$, excluding $\alpha_i$, with the unit type.
\end{definition}

\begin{definition}
  Given a type $\sigma$, in System $F_\kappa$, and a Petri net model 
  \begin{equation*}
  \mathcal{P}([\![\tau]\!]_n,\; [\![f]\!]_{n-1},\; [\![g]\!]_{n-1})
  \end{equation*}
  define
  \begin{equation*}
    \mathcal{P}_\alpha([\![\tau']\!]_n,\; [\![f]\!]_{n-1},\; [\![g]\!]_{n-1}) \text{ where } \tau'_i = \tau_i\{\alpha\}.
  \end{equation*}
\end{definition}

\end{document}
