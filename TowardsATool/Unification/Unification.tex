% !TeX root = Unification.tex
\documentclass[../../Dissertation.tex]{subfiles}

\begin{document}

\usetikzlibrary{arrows,shapes,automata,petri,cd}

\tikzset{
  place/.style={
    circle,
    thick,
    draw=black!75,
    fill=white!20,
    minimum size=6mm
  },
  transition/.style={
    rectangle,
    thick,
    fill=black,
    minimum width=8mm,
    inner ysep=2pt
  }
}

\subsection{A categorical semantics for unification}

\textbf{Definition 1.} Given a category $\mathbb{C}$ and a list $\alpha \in List\{+,-\}$, where $|\alpha| = n$, define 
\begin{flalign*}
\mathbb{C}^{\alpha} = \mathbb{C}^{\alpha_1} \times \mathbb{C}^{\alpha_2} \times ... \times \mathbb{C}^{\alpha_n},
\end{flalign*}\par
where $\mathbb{C}^+ = \mathbb{C}$ and $\mathbb{C}^- = \mathbb{C}^{op}$.\\\par

\textbf{Definition 2.} Given a list $\alpha \in List\{+,-\}$, define $\alpha^{+} = \alpha$ and $\alpha^{-}$ to be the list in\par which the sign of every element is flipped (i.e. every $+$ is switched to $-$ and vice versa).\\

\textbf{Definition 3.} Given a category $\mathbb{C}$, a list $\alpha \in List\{+,-\}$, and a functor $F : \mathbb{C}^\alpha \rightarrow \mathbb{C}$\par where $|\alpha| = n$, define 
\begin{flalign*}
F^{\alpha} = F^{\alpha_1} \times F^{\alpha_2} \times ... \times F^{\alpha_n},
\end{flalign*}\par
where $F^+ : \mathbb{C}^\alpha \rightarrow \mathbb{C} = F$ and $F^- : \mathbb{C}^{\alpha^-} \rightarrow \mathbb{C}^{op} = F^{op}$.\\\par

\textbf{Definition 4.} Given a list $q \in List\{+,-\}$, q is said to `match' a pattern\par $p \in List\{+,-\}$, $n = |p|$, with `substitution' $s \in List\{+,-\}$, if
\begin{flalign*}
q = s^{p_1} \mdoubleplus s^{p_2} \mdoubleplus ... \mdoubleplus s^{p_n}
\end{flalign*}\par
where $\mdoubleplus$ is the concatenation operator. Note that $n|s| = |q|$.\\

\textbf{Definition 5.} Let $\mathbb{C}$ be a category with binary products and $\alpha, \beta \in List\{+, -\}$ be\par lists, where $|\alpha| \leq |\beta|$. Given functors $F : \mathbb{C}^{\alpha} \rightarrow \mathbb{C},\ G: \mathbb{C}^{\beta} \rightarrow \mathbb{C}$, if there exists a\par `substitution' $s \in List\{+, -\}$ such that $\beta$ `matches' the `pattern' $\alpha$ with `substitution'\par $s$, $\left(n|\alpha| = |s||\alpha| = |\beta|\right)$, $F$ is said to `unify' with $G$ by a `substitution functor'\par $S : \mathbb{C}^{s} \rightarrow \mathbb{C}$, expressed as $F \sim G : S$. S is given by
\begin{flalign*}
\forall X_1, X_2, ..., X_n \in ob(\mathbb{C}).\ S(X_1, X_2, ... X_n) = X_1 \times X_2 \times ... \times X_n\\
\forall f_1, ..., f_n \in hom(\mathbb{C}).\ S(f_1, ..., f_n) = X_1 \times ... \times X_n \mapsto f_1 (X_1) \times ... \times f_n (X_n).
\end{flalign*}\par
Note the mixed variance in the domain of $S$; if $s_i = -$ then $f_i$ is a contravariant\par argument of the functor $S$. Observe that 
\begin{flalign*}
S (f_1 \circ g_1,...,f_n \circ g_n) = S (f_1,...,f_n) \circ S (g_1,...,g_n),\\
S (id_{X_1}, ..., id_{X_n}) = id_{S (X_1, ..., X_n)}.
\end{flalign*}

\textbf{Theorem 6.} Given lists $\alpha, \beta, \gamma \in List\{+,-\}$, let 
\begin{flalign*}
F : \mathbb{C}^\alpha \rightarrow \mathbb{C}, G : \mathbb{C}^\beta \rightarrow \mathbb{C}, S : \mathbb{C}^\gamma \rightarrow \mathbb{C}, 
\end{flalign*}\par
be functors. If $F \sim G : S$, then $F \cdot S^\alpha \cong G$ (this both needs to be proved and better\par expressed, since the codomain of F does not exactly match the domain of S).\\

\textbf{Definition 7.} Given a category $\mathbb{C}$, lists $\alpha, \beta, \gamma \in List\{+,-\}$, functors
\begin{flalign*}
F : \mathbb{C}^{\gamma} \rightarrow \mathbb{C}, G: \mathbb{C}^{\alpha} \rightarrow \mathbb{C}, H : \mathbb{C}^{\beta} \rightarrow \mathbb{C},
\end{flalign*}\par
and a dinatural transformation
\begin{flalign*}
\phi : G \rightarrow H,
\end{flalign*}\par
the left whiskering of $\phi$ with F is given by
\begin{flalign*}
(\phi * F)_X : G (F^{\alpha_1}(X), ..., F^{\alpha_n}(X)) \rightarrow H (F^{\beta_1}(X), ..., F^{\beta_n}(X)) = \phi_{F(X)},
\end{flalign*}\par
where $F^+ = F$ and $F^- = F^{op}$.\\\par

\textbf{Theorem 8.} Given a category $\mathbb{C}$, lists $\alpha, \beta, \gamma \in List\{+,-\}$, functors
\begin{flalign*}
F : \mathbb{C}^{\gamma} \rightarrow \mathbb{C}, G: \mathbb{C}^{\alpha} \rightarrow \mathbb{C}, H : \mathbb{C}^{\beta} \rightarrow \mathbb{C},
\end{flalign*}\par
and a dinatural transformation
\begin{flalign*}
\phi : G \rightarrow H,
\end{flalign*}\par
$\phi * F$ is dinatural. Dinatural transformations are therefore closed under left whiskering.\par (Proof to be added)\\

\textbf{Definition 9.} Given a cartesian closed category $\mathbb{C}$, lists $\alpha, \beta, \alpha', \beta' \in List\{+,-\}$,\par functors
\begin{flalign*}
F : \mathbb{C}^{\alpha} \rightarrow \mathbb{C}, G: \mathbb{C}^{\beta} \rightarrow \mathbb{C},\\
H : \mathbb{C}^{\alpha'} \rightarrow \mathbb{C}, K: \mathbb{C}^{\beta'} \rightarrow \mathbb{C},
\end{flalign*}\par
and dinatural transformations
\begin{flalign*}
\phi : F \rightarrow G, \psi : H \rightarrow K,
\end{flalign*}\par
if $G \sim H : S$ then $\psi$ is composable with $\phi * S$; the domain of $\psi$ is naturally isomorphic\par to the codomain of $\phi * S$ (5). Composition is given by
\begin{flalign*}
\psi \cdot \phi = \psi \circ (\phi * S).
\end{flalign*}\par
Conversely, if $H \sim G: S'$, composition is given by
\begin{flalign*}
\psi \cdot \phi = (\psi * S') \circ \phi.
\end{flalign*}\par
Note that $\psi \cdot \phi$ is not necessarily a dinatural transformation.\\

\textbf{Definition 10.} Let $[n] = \{i \in \mathbb{N} : i \leq n\}$ and $F: \mathbb{C}^\alpha \rightarrow \mathbb{C}$ and $G: \mathbb{C}^\beta \rightarrow \mathbb{C}$\par be functors. The type of a dinatural transformation (McCusker and Santamaria 2018), 
\par$\phi : F \rightarrow G$, is given by three sets $[\alpha], [\beta], [k]$, and two functions $\sigma : [\alpha] \rightarrow [k],$\par$\tau : [\beta] \rightarrow [k]$ and is denoted $f = \cospan{[\alpha]}{\sigma}{[k]}{\tau}{[\beta]}$. Given a type $f$, $\phi$ is given to be a\par family of morphisms
\begin{flalign*}
\left(\phi_{A_1,...,A_k} : F(A_{\sigma(1)},...,A_{\sigma(|\alpha|)}) \rightarrow G(A_{\tau(1)},...,A_{\tau(|\beta|)})\right)_{A_1 \times...\times A_k \in \mathbb{C}^k}.
\end{flalign*}\par
The functions $\sigma$ and $\tau$ are said to determine which arguments of $F$ and $G$ are to be\par equated.\\

\textbf{Definition 11.} The construction of a type given in (8) is a familiar one; it is a cospan\par in the category $Set$. McCusker and Santamaria construct the category $\mathbb{T}$ypes of\par transformation types, where $ob(\mathbb{T}\text{ypes}) = \{[i] : i \in \mathbb{N}\}$ and $\forall f : n \rightarrow m$, $f$ is given by\par a cospan of the form $\cospan{n}{\sigma}{k}{\tau}{m}$. Composition of a morphism f and $g = \cospan{m}{\sigma'}{p}{\tau'}{t}$,\par $g \cdot f = \cospan{n}{\zeta \cdot \sigma}{q}{\xi \cdot \tau'}{t}$ is given by calculating the pushout, 
\begin{flalign*}
(q \in ob(\mathbb{T}\text{ypes}),\ \xi : p \rightarrow q,\ \zeta : k \rightarrow q),
\end{flalign*}\par
of $\tau$ against $\sigma'$.
\\\par
This pushout is calculated in the usual way for sets, where $q$ is given to be quotient of\par the disjoint union of $k$ and $p$ by the equivalence relation $\tau(x) \sim \sigma'(x)$. Alternatively,\par the construction of this pushout can be described as the disjoint union of $p$ and $k$ in\par which any elements obtained by applying functions $\tau$ and $\sigma'$, respectively, to the same\par value from $m$, are paired.\\

\textbf{Definition 12.} Given a category $\mathbb{C}$, lists $\alpha, \beta, \gamma \in List\{+,-\}$, functors 
\begin{flalign*}
F : \mathbb{C}^\gamma \rightarrow \mathbb{C}, G : \mathbb{C}^\alpha \rightarrow \mathbb{C}, H : \mathbb{C}^\beta \rightarrow \mathbb{C},
\end{flalign*}\par
and a dinatural transformation $\phi : G \rightarrow H$, of type $\cospan{[\alpha]}{\sigma}{[k]}{\tau}{[\beta]}$, the type of\par $\phi * F$ is given by $\cospan{[\alpha * \gamma]}{\sigma'}{[k * \gamma]}{\tau'}{[\beta * \gamma]}$, where 
\begin{flalign*}
\sigma' = \bigcup\limits_{i=1}^{|\alpha|} \{(|\gamma| * (i - 1) + j,\ \sigma(i) + j - 1)\ |\ j \in \gamma \}\\
\tau' = \bigcup\limits_{i=1}^{|\beta|} \{(|\gamma| * (i - 1) + j,\ \tau(i) + j - 1)\ |\ j \in \gamma \}.
\end{flalign*}\par

\end{document}
