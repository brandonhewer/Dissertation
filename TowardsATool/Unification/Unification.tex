% !TeX root = Unification.tex
\documentclass[../../Dissertation.tex]{subfiles}

\begin{document}

\usetikzlibrary{arrows,shapes,automata,petri,cd}

\tikzset{
  place/.style={
    circle,
    thick,
    draw=black!75,
    fill=white!20,
    minimum size=6mm
  },
  transition/.style={
    rectangle,
    thick,
    fill=black,
    minimum width=8mm,
    inner ysep=2pt
  }
}

\subsection{A categorical semantics for unification}
{
\noindent
\textbf{Definition 1.} Given a category $\mathbb{C}$ and a list $\alpha \in List\{+,-\}$, where $|\alpha| = n$, define 
\begin{flalign*}
\mathbb{C}^{\alpha} = \mathbb{C}^{\alpha_1} \times \mathbb{C}^{\alpha_2} \times ... \times \mathbb{C}^{\alpha_n},
\end{flalign*}
where $\mathbb{C}^+ = \mathbb{C}$ and $\mathbb{C}^- = \mathbb{C}^{op}$.\\
}

{
\noindent
\textbf{Definition 2.} Given a list $\alpha \in List\{+,-\}$, define $\alpha^{+} = \alpha$ and $\alpha^{-}$ to be the list in which the sign of every element is flipped (i.e. every $+$ is switched to $-$ and vice versa).\\
}

{
\noindent
\textbf{Definition 3.} Given a category $\mathbb{C}$, a list $\alpha \in List\{+,-\}$, and a functor $F : \mathbb{C}^\alpha \rightarrow \mathbb{C}$ where $|\alpha| = n$, define 
\begin{flalign*}
F^{\alpha} = F^{\alpha_1} \times F^{\alpha_2} \times ... \times F^{\alpha_n},
\end{flalign*}
where $F^+ : \mathbb{C}^\alpha \rightarrow \mathbb{C} = F$ and $F^- : \mathbb{C}^{\alpha^-} \rightarrow \mathbb{C}^{op} = F^{op}$.\\
}

{
\noindent
\textbf{Definition 4.} Given a category $\mathbb{C}$, lists $\alpha, \beta, \gamma \in List\{+,-\}$, functors
\begin{flalign*}
F : \mathbb{C}^{\gamma} \rightarrow \mathbb{C},\ G: \mathbb{C}^{\alpha} \rightarrow \mathbb{C},\ H : \mathbb{C}^{\beta} \rightarrow \mathbb{C},
\end{flalign*}
and a dinatural transformation
\begin{flalign*}
\phi : G \rightarrow H,
\end{flalign*}
the left whiskering of $\phi$ with F is given by
\begin{flalign*}
(\phi * F)_X : G (F^{\alpha_1}(X), ..., F^{\alpha_n}(X)) \rightarrow H (F^{\beta_1}(X), ..., F^{\beta_n}(X)) = \phi_{F(X)},
\end{flalign*}
where $F^+ = F$ and $F^- = F^{op}$.\\
}

{
\noindent
\textbf{Theorem 5.} Given a category $\mathbb{C}$, lists $\alpha, \beta, \gamma \in List\{+,-\}$, functors
\begin{flalign*}
F : \mathbb{C}^{\gamma} \rightarrow \mathbb{C}, G: \mathbb{C}^{\alpha} \rightarrow \mathbb{C}, H : \mathbb{C}^{\beta} \rightarrow \mathbb{C},
\end{flalign*}
and a dinatural transformation
\begin{flalign*}
\phi : G \rightarrow H,
\end{flalign*}
$\phi * F$ is dinatural. Dinatural transformations are therefore closed under left whiskering. (Proof to be added)\\
}

{
\noindent
\textbf{Definition 6.} Given a list $q \in List\{+,-\}$, q is said to `match' a pattern $p \in List\{+,-\}$, $n = |p|$, with `substitution' $s \in List\{+,-\}$, denoted $q = p(s)$, if
\begin{flalign*}
q = s^{p_1} \mdoubleplus s^{p_2} \mdoubleplus ... \mdoubleplus s^{p_n}
\end{flalign*}
where $\mdoubleplus$ is the concatenation operator. Note that $n|s| = |q|$.\\
}

{
\noindent
\textbf{Definition 7.} Let $\mathbb{C}$ be a category and $\alpha$, $\beta \in List\{+, -\}$ be lists, where $|\alpha| \leq |\beta|$. Given functors
\begin{flalign*}
F : \mathbb{C}^{\alpha} \rightarrow \mathbb{C},\ G: \mathbb{C}^{\beta} \rightarrow \mathbb{C},
\end{flalign*}
if there exists $s \in List\{+, -\}$ such that 
\begin{flalign*}
\beta = \alpha(s),\ n|\alpha| = |s||\alpha| = |\beta|, 
\end{flalign*}
and there exists a functor $S : \mathbb{C}^s \rightarrow \mathbb{C}$ such that
\begin{flalign*}
G \cdot S^\beta \cong F, 
\end{flalign*}
$F$ is said to `unify' with $G$ by means of the `substitution functor' $S$, expressed as $F \sim G : S$.\\
}

{
\noindent
\textbf{Definition 8.} Given a cartesian closed category $\mathbb{C}$, lists $\alpha, \beta, \alpha', \beta' \in List\{+,-\}$, functors
\begin{flalign*}
F : \mathbb{C}^{\alpha} \rightarrow \mathbb{C}, G: \mathbb{C}^{\beta} \rightarrow \mathbb{C},\\
H : \mathbb{C}^{\alpha'} \rightarrow \mathbb{C}, K: \mathbb{C}^{\beta'} \rightarrow \mathbb{C},
\end{flalign*}
and dinatural transformations
\begin{flalign*}
\phi : F \rightarrow G, \psi : H \rightarrow K,
\end{flalign*}
if $G \sim H : S$ then $\psi$ is composable with $\phi * S$; the domain of $\psi$ is naturally isomorphic to the codomain of $\phi * S$ (5). Composition is given by
\begin{flalign*}
\psi \cdot \phi = \psi \circ (\phi * S).
\end{flalign*}
Conversely, if $H \sim G: S'$, composition is given by
\begin{flalign*}
\psi \cdot \phi = (\psi * S') \circ \phi.
\end{flalign*}
Note that $\psi \cdot \phi$ is not necessarily a dinatural transformation.\\
}

{
\noindent
\textbf{Definition 9.} Let $[n] = \{i \in \mathbb{N} : i \leq n\}$ and $F: \mathbb{C}^\alpha \rightarrow \mathbb{C}$ and $G: \mathbb{C}^\beta \rightarrow \mathbb{C}$ be functors. The type of a dinatural transformation (McCusker and Santamaria 2018), 
$\phi : F \rightarrow G$, is given by three sets $[\alpha], [\beta], [k]$, and two functions $\sigma : [\alpha] \rightarrow [k]$, $\tau : [\beta] \rightarrow [k]$ and is denoted $f = \cospan{[\alpha]}{\sigma}{[k]}{\tau}{[\beta]}$. Given a type $f$, $\phi$ is given to be a family of morphisms
\begin{flalign*}
\left(\phi_{A_1,...,A_k} : F(A_{\sigma(1)},...,A_{\sigma(|\alpha|)}) \rightarrow G(A_{\tau(1)},...,A_{\tau(|\beta|)})\right)_{A_1 \times...\times A_k \in \mathbb{C}^k}.
\end{flalign*}
The functions $\sigma$ and $\tau$ are said to determine which arguments of $F$ and $G$ are to be equated.\\
}

{
\noindent
\textbf{Definition 10.} The construction of a type given in (8) is a familiar one; it is a cospan in the category $Set$. McCusker and Santamaria construct the category $\mathbb{T}$ypes of transformation types, where $ob(\mathbb{T}\text{ypes}) = \{[i] : i \in \mathbb{N}\}$ and $\forall f : n \rightarrow m$, $f$ is given by a cospan of the form $\cospan{n}{\sigma}{k}{\tau}{m}$. Composition of a morphism f and $g = \cospan{m}{\sigma'}{p}{\tau'}{t}$, $g \cdot f = \cospan{n}{\zeta \cdot \sigma}{q}{\xi \cdot \tau'}{t}$ is given by calculating the pushout, 
\begin{flalign*}
(q \in ob(\mathbb{T}\text{ypes}),\ \xi : p \rightarrow q,\ \zeta : k \rightarrow q),
\end{flalign*}
of $\tau$ against $\sigma'$. This pushout is calculated in the usual way for sets, where $q$ is given to be quotient of the disjoint union of $k$ and $p$ by the equivalence relation $\tau(x) \sim \sigma'(x)$.\\
}

{
\noindent
\textbf{Definition 11.} Given a category $\mathbb{C}$, lists $\alpha, \beta, \gamma \in List\{+,-\}$, functors 
\begin{flalign*}
F : \mathbb{C}^\gamma \rightarrow \mathbb{C}, G : \mathbb{C}^\alpha \rightarrow \mathbb{C}, H : \mathbb{C}^\beta \rightarrow \mathbb{C},
\end{flalign*}
and a dinatural transformation $\phi : G \rightarrow H$, of type $\cospan{[\alpha]}{\sigma}{[k]}{\tau}{[\beta]}$, the type of $\phi * F$ is given by $\cospan{[\alpha * \gamma]}{\sigma'}{[k * \gamma]}{\tau'}{[\beta * \gamma]}$, where 
\begin{flalign*}
\sigma' = \bigcup\limits_{i=1}^{|\alpha|} \{(|\gamma| * (i - 1) + j,\ \sigma(i) + j - 1)\ |\ j \in \gamma \}\\
\tau' = \bigcup\limits_{i=1}^{|\beta|} \{(|\gamma| * (i - 1) + j,\ \tau(i) + j - 1)\ |\ j \in \gamma \}.
\end{flalign*}
}

\end{document}
