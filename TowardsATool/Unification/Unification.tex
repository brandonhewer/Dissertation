% !TeX root = Unification.tex
\documentclass[../../Dissertation.tex]{subfiles}

\begin{document}

\usetikzlibrary{arrows,shapes,automata,petri,cd}

\tikzset{
  place/.style={
    circle,
    thick,
    draw=black!75,
    fill=white!20,
    minimum size=6mm
  },
  transition/.style={
    rectangle,
    thick,
    fill=black,
    minimum width=8mm,
    inner ysep=2pt
  }
}

\subsection{Categorical semantics for System $F_\kappa$}
Recall that Petri nets in the desired computational tool correspond to dinatural coherence conditions. The intended system should provide a means to compose types, and present the coherence condition of the composite type as a Petri net. Unification and substitution are required for composition in System $F_\kappa$. Accordingly, it is necessary to understand and formulate the categorical semantics underlying unification and substitution. 
\par
Furthermore, depicting conditions in which not all occurrences of a type variable are equated, requires understanding McCusker and Santamaria's notion of a naturality type. In addition, the ramifications of unification and substitution on naturality types, will need to be determined. This will facilitate a categorical semantics for System $F_\kappa$.

\begin{proposition}\label{props:constant}
For every type $\tau$, in System $F_\kappa$, there exists a constant type constructor $\mathcal{C}_\tau : * \Rightarrow *$ given by $C (\sigma) = \tau$. To prove $C_\tau$ is a type constructor, in System $F_\kappa$, it is necessary to show the existence of a polymorphic function
\begin{equation*}
\mathcal{F}_{C_\tau} : (\alpha \rightarrow \beta) \rightarrow C_\tau (\alpha) \rightarrow C_\tau (\beta),
\end{equation*}
which satisfies the functorial laws. This is achieved by defining 
\begin{equation*}
\mathcal{F}_{C_\tau}(f) = \lambda x^\tau.x. 
\end{equation*}
It is routine to prove both laws are satisfied.
\end{proposition}

\begin{proposition}\label{props:id}
In System $F_\kappa$, there exists an identity type constructor $Id : * \Rightarrow *$, given by $Id(\tau) = \tau$. To prove $Id$ is a type constructor, in System $F_\kappa$, it is necessary to show the existence of a polymorphic function
\begin{equation*}
\mathcal{F}_{Id} : (\alpha \rightarrow \beta) \rightarrow Id (\alpha) \rightarrow Id (\beta),
\end{equation*}
which satisfies the functorial laws. This is achieved by defining 
\begin{equation*}
\mathcal{F}_{Id}(f) = f
\end{equation*}
It is routine to prove both laws are satisfied.
\end{proposition}

\begin{proposition}\label{props:arrow}
In System $F_\kappa$, $\rightarrow$ is a type constructor of the form 
\begin{flalign*}
(\rightarrow) : *^- \Rightarrow * \Rightarrow *.
\end{flalign*}
Observe the application of $\rightarrow$ on types:
\begin{flalign*}
(\rightarrow)(\tau_1,\ \tau_2) = \tau_1 \rightarrow \tau_2.
\end{flalign*}
To prove $\rightarrow$ is a type constructor in System $F_\kappa$, it is necessary to show there exists a polymorphic function
\begin{flalign*}
  \mathcal{F}_\rightarrow : (\alpha_1 \rightarrow \alpha_2) \rightarrow (\beta_1 \rightarrow \beta_2) \rightarrow (\alpha_2 \rightarrow \beta_1) \rightarrow (\alpha_1 \rightarrow \beta_2),
\end{flalign*}
such that the functorial laws, outlined in Definition 6, are satisfied. Given terms 
\begin{flalign*}
f : \sigma_2 \rightarrow \tau_2,\ g : \sigma_1 \rightarrow \sigma_2,\ h : \tau_1 \rightarrow \sigma_1,
\end{flalign*}  
$\mathcal{F}_\rightarrow$ is defined as
\begin{flalign*}
  \mathcal{F}_\rightarrow(h, f) = \lambda g. f \circ g \circ h.
\end{flalign*}  
The identity law is shown to be satisfied by
\begin{flalign*}
  \mathcal{F}_\rightarrow(\iota_{\sigma_1}, \iota_{\sigma_2}) = \lambda g. \iota_{\sigma_2} \circ g \circ \iota_{\sigma_1} = \lambda g. g.
\end{flalign*}  
Given terms $f' : \tau_2 \rightarrow \rho_2$, $g' : \tau_1 \rightarrow \tau_2$, $h' : \rho_1 \rightarrow \tau_1$,
\begin{flalign*}
  \mathcal{F}_\rightarrow(h', f') \circ \mathcal{F}_\rightarrow(h, f) &= (\lambda g'. f' \circ g' \circ h') \circ (\lambda g. f \circ g \circ h)\\
  &= \lambda g. f' \circ (f \circ g \circ h) \circ h'\\
  &= \lambda g. (f' \circ f) \circ g \circ (h \circ h')\\
  &= \mathcal{F}_\rightarrow(h \circ h', f' \circ f),
\end{flalign*} 
thus, as expected, composition is preserved.
\end{proposition}

\begin{proposition}\label{prop:times}
  In System $F_\kappa$, $\times$ is a type constructor of the form
  \begin{equation*}
    (\times) : * \Rightarrow * \Rightarrow *.
  \end{equation*}
  Observe the application of $\times$ on types:
  \begin{equation*}
    (\times)(\tau_1,\ \tau_2) = \tau_1 \times \tau_2.
  \end{equation*}
  To prove $\times$ is a type constructor in System $F_\kappa$, it is necessary to show there exists a polymorphic function
  \begin{equation*}
    \mathcal{F}_\times : (\alpha_1 \rightarrow \alpha_2) \rightarrow (\beta_1 \rightarrow \beta_2) \rightarrow (\alpha_1 \times \beta_1) \rightarrow (\alpha_2 \times \beta_2),
  \end{equation*}
  such that the functorial laws, outlined in Definition 6, are satisfied. Given terms 
  \begin{equation*}
    f : \tau_1 \rightarrow \sigma_1,\ g : \tau_2 \rightarrow \sigma_2,\ p : \tau_1 \times \tau_2, 
  \end{equation*}
  $\mathcal{F}_\times$ is defined as
  \begin{equation*}
    \mathcal{F}_\times(f, g) = p \rightarrow \langle (f \circ \pi_1)\ p,\ (g \circ \pi_2)\ p \rangle.
  \end{equation*} 
  The identity law is shown to be satisfied by
  \begin{flalign*}
    \mathcal{F}_\times(\iota_{\tau_1}, \iota_{\tau_2}) &= \lambda p. \langle (\iota_{\tau_1} \circ \pi_1)\ p,\ (\iota_{\tau_2} \circ \pi_2)\ p \rangle \\&= \lambda p. \langle \pi_1\ p,\ \pi_2\ p \rangle \\&= \lambda p. p.
  \end{flalign*}
Given terms $f' : \sigma_1 \rightarrow \mu_1$, $g' : \sigma_2 \rightarrow \mu_2$, $p' : \sigma_1 \times \sigma_2$,
\begin{flalign*}
  \mathcal{F}_\times(f',g') \circ \mathcal{F}_\times(f,g) &= (\lambda p'. \langle (f' \circ \pi_1)\ p',\ (g' \circ \pi_2)\ p' \rangle)\ \circ \\
  &\quad\ (\lambda p. \langle (f \circ \pi_1)\ p,\ (g \circ \pi_2)\ p \rangle)\\
  &= \lambda p. \langle (f' \circ f \circ \pi_1)\ p,\ (g' \circ g \circ \pi_2)\ p \rangle\\
  &= \mathcal{F}_\times(f' \circ f, g' \circ g),
\end{flalign*}
thus, as expected, composition is preserved.
\end{proposition}

\begin{definition}
Let $\theta$, $\gamma \in List\{+,-\}$ be lists, where $|\theta| = n$ and $|\gamma| = m$. Given kinds $\upsilon_1^\theta$, $\upsilon_2^\gamma$, and type constructors $T_1 : \upsilon_1^\theta$ and $T_2 : \upsilon_2^\gamma$, the type constructor
\begin{equation*}
  T_1(T_2) : \omega^{([\theta_2,\ \theta_3,\ \ldots,\ \theta_n]\ \mdoubleplus\ \gamma^{\theta_1})}
\end{equation*}
is given by
\begin{equation*}
  T_1(T_2)\left(\sigma_2,\ \ldots,\ \sigma_n,\ \tau_1,\ \ldots,\ \tau_m) = T_1(T_2(\tau_1,\ \ldots,\ \tau_m),\ \sigma_2,\ \ldots,\ \sigma_n\right).
\end{equation*}
The proof that $T_1(T_2)$ is a type constructor in System $F_\kappa$ is given by defining $\mathcal{F}_{T_1(T_2)}$ as
\begin{flalign*}
  \mathcal{F}_{T_1(T_2)}(f_2,\ \ldots,\ f_n,\ g_1,\ \ldots,\ g_m) = \mathcal{F}_{T_1}(\mathcal{F}_{T_2}(g_1,\ \ldots,\ g_m),\ f_2,\ \ldots,\ f_n).
\end{flalign*}
Multiple applications of this operation, such as $T_1(T_2)(T_3)$, can be expressed uncurried as $T_1(T_2,\ T_3)$. 
\end{definition}

\begin{example}
  Recall the definitions for the functors $\rightarrow$ and $\times$, outlined in propositions \ref{props:arrow} and  \ref{prop:times}, respectively. The type constructor $\rightarrow(\times)$ is given by
  \begin{equation*}
    (\rightarrow(\times))(\gamma,\ \alpha,\ \beta) = \alpha \times \beta \rightarrow \gamma.
  \end{equation*}
  A second example, $\rightarrow(\rightarrow, \rightarrow)$, is given by
  \begin{equation*}
    (\rightarrow(\rightarrow, \rightarrow))(\alpha,\ \alpha',\ \beta,\ \beta') = (\alpha \rightarrow \alpha') \rightarrow \beta \rightarrow \beta'.
  \end{equation*}
\end{example}

\begin{definition}
  Let $\theta$ be a list, where $|\theta| = n$, and $\upsilon^\theta$ be a kind, in System $F_\kappa$. A type $\tau$ is said to have a higher-kinded representation if there exists a type constructor $T : \upsilon^\theta$ and there exists types $\sigma_1,\ \sigma_2,\ \ldots,\ \sigma_n$, such that
  \begin{flalign*}
    T\ \sigma_1\ \sigma_2\ \ldots\ \sigma_n = \tau.
  \end{flalign*}
  $T$ is said to be a higher-kinded representation of $\tau$.
\end{definition}

\begin{example}
  Given a type $\tau \times \sigma \rightarrow \tau' \times \sigma'$, the type constructor $\rightarrow(\times, \times)$ is a corresponding higher-kinded representation. The condition is satisfied by
  \begin{equation*}
    (\rightarrow(\times, \times))(\tau,\ \sigma,\ \tau',\ \sigma') = \tau \times \sigma \rightarrow \tau' \times \sigma'.
  \end{equation*}
\end{example}

\begin{theorem}\label{th:kindrep}
Recall the definitions for the functors $Id$, $\rightarrow$, $\times$ outlined in propositions \ref{props:constant}, \ref{props:id}, \ref{props:arrow} and \ref{prop:times}, respectively. It follows that for every type $\tau$ in System $F_\kappa$, there exists a higher-kinded representation of $\tau$, denoted $\mathcal{K}(\tau)$, given by
\begin{flalign*}
  \mathcal{K}(c) &= c,\\
  \mathcal{K}(\alpha) &= Id,\\
  \mathcal{K}(\tau_1 \times \tau_2) &=  (\times)(\mathcal{K}(\tau_1),\ \mathcal{K}(\tau_2))\\
  \mathcal{K}(\tau_1 \rightarrow \tau_2) &= (\rightarrow)(\mathcal{K}(\tau_1),\ \mathcal{K}(\tau_2))\\
  \mathcal{K}(T\ \tau_1\ \tau_2\ \ldots\ \tau_n) &= T (\mathcal{K}(\tau_1),\ \mathcal{K}(\tau_2),\ \ldots,\ \mathcal{K}(\tau_n))\\
  \mathcal{K}(F\ \tau_1\ \tau_2\ \ldots\ \tau_n) &= F (\mathcal{K}(\tau_1),\ \mathcal{K}(\tau_2),\ \ldots,\ \mathcal{K}(\tau_n)),
\end{flalign*}
where $T$ is a type constructor and $F$ is a polymorphic type constructor, both with $n$ arguments. Note that $\mathcal{M}(\tau)$ may be of the kind $*$, if $\tau$ contains no type variables. As such, there exists a type constructor for any type $\tau$, denoted $\mathcal{M}(\tau)$, given by
\begin{flalign*}
  \mathcal{M}(\tau) &=
  \begin{cases}
    T & \text{if } |\theta| > 0\\
    \mathcal{C}_T & \text{otherwise}
  \end{cases}\\
  &\qquad \text{where } (T : \upsilon^\theta) = \mathcal{K}(\tau).
\end{flalign*}
\end{theorem}

\begin{example}
  Let $\mathbb{N}$ be a primitive type. Given a type 
  \begin{equation*}
    \tau = \forall \alpha.(\alpha \rightarrow \mathbb{N}),
  \end{equation*}
  there exists a corresponding higher-kinded representation $\mathcal{M}(\tau)$, given by
  \begin{equation*}
    (\rightarrow(Id,\ \mathbb{N}))(\alpha) = \alpha \rightarrow \mathbb{N}.
  \end{equation*}
  Similarly, given a type
  \begin{equation*}
    \sigma = \forall \alpha.\forall \beta.\alpha \times \beta,
  \end{equation*}
  there exists a higher-kinded representation $\mathcal{M}(\sigma)$, given by
  \begin{equation*}
    (\times(Id,Id))(\alpha,\ \beta) = \alpha \times \beta.
  \end{equation*}
  Finally, given a more complex type
  \begin{equation*}
    \rho = \forall \alpha.\ \forall \beta.(\alpha \rightarrow \mathbb{N}) \rightarrow \alpha \times \beta,
  \end{equation*}
  there exists a higher-kinded representation $\mathcal{M}(\rho)$, given by
  \begin{equation*}
    (\rightarrow(\rightarrow(Id, \mathbb{N}),\times(Id,Id)))(\alpha,\ \beta) = (\alpha \rightarrow \mathbb{N}) \rightarrow \alpha \times \beta.
  \end{equation*}
\end{example}

\begin{definition}
Given a category $\mathbb{C}$ and a list $\theta \in List\{+,-\}$, where $|\theta| = n$, define 
\begin{flalign*}
\mathbb{C}^{\theta} = \mathbb{C}^{\theta_1} \times \mathbb{C}^{\theta_2} \times ... \times \mathbb{C}^{\theta_n},
\end{flalign*}
where $\mathbb{C}^+ = \mathbb{C}$ and $\mathbb{C}^- = \mathbb{C}^{op}$.
\end{definition}

\begin{corollary}\label{cor:functor}
Let $\theta \in List\{+,-\}$ be a list, and $\upsilon^\theta$ be a higher-kinded type in System $F_\kappa$. Recall that Theorem \ref{th:kindrep} states that given any type $\tau$, there exists a type constructor $T : \upsilon^\theta = \mathcal{M}(\tau)$. Given the category $\mathbb{C}$, of System $F_\kappa$ types, it follows that there exists a functor representation $\mathcal{R}(\tau) : \mathbb{C}^\theta \rightarrow \mathbb{C}$. $\mathcal{R}(\tau)$ operates on types identically to $T$, and operates on terms identically to $\mathcal{F}_T$. The functoriality of $\mathcal{R}(\tau)$ is given by the definition of $\mathcal{F}_T$. In subsequent definitions, the notation $\mathcal{V}(\tau)$ will be used to denote the list $\theta$, i.e., the list of variances.
\end{corollary}

\begin{remark}
  An approach for identifying functors from types in System $F_\kappa$ is formulated in Corollary \ref{cor:functor}. As a consequence, the remaining definitions in this section, pertaining to a categorical semantics for composition in System $F_\kappa$, will be expressed in terms of the functors involved.
\end{remark}

\begin{definition}
Given a category $\mathbb{C}$, a list $\alpha \in List\{+,-\}$, and a functor $F : \mathbb{C}^\alpha \rightarrow \mathbb{C}$ where $|\alpha| = n$, define 
\begin{flalign*}
F^{\alpha} = F^{\alpha_1} \times F^{\alpha_2} \times ... \times F^{\alpha_n},
\end{flalign*}
where $F^+ : \mathbb{C}^\alpha \rightarrow \mathbb{C} = F$ and $F^- : \mathbb{C}^{\alpha^-} \rightarrow \mathbb{C}^{op} = F^{op}$.
\end{definition}

\begin{definition}
Given a category $\mathbb{C}$, lists $\alpha, \beta, \gamma \in List\{+,-\}$, functors
\begin{flalign*}
F : \mathbb{C}^{\gamma} \rightarrow \mathbb{C},\ G: \mathbb{C}^{\alpha} \rightarrow \mathbb{C},\ H : \mathbb{C}^{\beta} \rightarrow \mathbb{C},
\end{flalign*}
and a dinatural transformation
\begin{flalign*}
\phi : G \rightarrow H,
\end{flalign*}
the left whiskering of $\phi$ with F is given by
\begin{flalign*}
(\phi * F)_X : G (F^{\alpha_1}(X), ..., F^{\alpha_n}(X)) \rightarrow H (F^{\beta_1}(X), ..., F^{\beta_n}(X)) = \phi_{F(X)},
\end{flalign*}
where $F^+ = F$ and $F^- = F^{op}$.
\end{definition}

\begin{definition}\label{def:catunify}
Let $\mathbb{C}$ be a category and $\alpha$, $\beta \in List\{+, -\}$ be lists, where $|\alpha| \leq |\beta|$. Given functors
\begin{flalign*}
F : \mathbb{C}^{\alpha} \rightarrow \mathbb{C},\ G: \mathbb{C}^{\beta} \rightarrow \mathbb{C},
\end{flalign*}
if there exists $s \in List\{+, -\}$ such that 
\begin{flalign*}
\beta = \alpha(s),\ n|\alpha| = |s||\alpha| = |\beta|, 
\end{flalign*}
and there exists a functor $S : \mathbb{C}^s \rightarrow \mathbb{C}$ such that
\begin{flalign*}
G \cdot S^\beta \cong F, 
\end{flalign*}
$F$ is said to `unify' with $G$ by means of the `substitution functor' $S$, expressed as $F \sim G : S$.
\end{definition}

\begin{definition}
Given a category $\mathbb{C}$, lists $\alpha, \beta, \alpha', \beta' \in List\{+,-\}$, functors
\begin{flalign*}
F : \mathbb{C}^{\alpha} \rightarrow \mathbb{C}, G: \mathbb{C}^{\beta} \rightarrow \mathbb{C},\\
H : \mathbb{C}^{\alpha'} \rightarrow \mathbb{C}, K: \mathbb{C}^{\beta'} \rightarrow \mathbb{C},
\end{flalign*}
and dinatural transformations
\begin{flalign*}
\phi : F \rightarrow G, \psi : H \rightarrow K,
\end{flalign*}
if $G \sim H : S$ then $\psi$ is composable with $\phi * S$; the domain of $\psi$ is naturally isomorphic to the codomain of $\phi * S$ (Definition \ref{def:catunify}). Composition is given by
\begin{flalign*}
\psi \cdot \phi = \psi \circ (\phi * S).
\end{flalign*}
Conversely, if $H \sim G: S'$, composition is given by
\begin{flalign*}
\psi \cdot \phi = (\psi * S') \circ \phi.
\end{flalign*}
Note that $\psi \cdot \phi$ is not necessarily a dinatural transformation.
\end{definition}

\begin{definition}\label{def:nattype}
Let $[n] = \{i \in \mathbb{N} : i \leq 
n\}$ and $F: \mathbb{C}^\alpha \rightarrow \mathbb{C}$ and $G: \mathbb{C}^\beta \rightarrow \mathbb{C}$ be functors. The type of a dinatural transformation (McCusker and Santamaria 2018), 
$\phi : F \rightarrow G$, is given by three sets $[\alpha], [\beta], [k]$, and two functions $\sigma : [\alpha] \rightarrow [k]$, $\tau : [\beta] \rightarrow [k]$ and is denoted $f = \cospan{[\alpha]}{\sigma}{[k]}{\tau}{[\beta]}$. Given a type $f$, $\phi$ is given to be a family of morphisms
\begin{flalign*}
\left(\phi_{A_1,...,A_k} : F(A_{\sigma(1)},...,A_{\sigma(|\alpha|)}) \rightarrow G(A_{\tau(1)},...,A_{\tau(|\beta|)})\right)_{A_1 \times...\times A_k \in \mathbb{C}^k}.
\end{flalign*}
The functions $\sigma$ and $\tau$ are said to determine which arguments of $F$ and $G$ are to be equated.
\end{definition}

\begin{definition}
The construction of a type given in Definition \ref{def:nattype} is a familiar one; it is a cospan in the category $Set$. McCusker and Santamaria construct the category $\mathbb{T}$ypes of transformation types, where $ob(\mathbb{T}\text{ypes}) = \{[i] : i \in \mathbb{N}\}$ and $\forall f : n \rightarrow m$, $f$ is given by a cospan of the form $\cospan{n}{\sigma}{k}{\tau}{m}$. Composition of a morphism f and $g = \cospan{m}{\sigma'}{p}{\tau'}{t}$, $g \cdot f = \cospan{n}{\zeta \cdot \sigma}{q}{\xi \cdot \tau'}{t}$ is given by calculating the pushout, 
\begin{flalign*}
(q \in ob(\mathbb{T}\text{ypes}),\ \xi : p \rightarrow q,\ \zeta : k \rightarrow q),
\end{flalign*}
of $\tau$ against $\sigma'$. This pushout is calculated in the usual way for sets, where $q$ is given to be quotient of the disjoint union of $k$ and $p$ by the equivalence relation $\tau(x) \sim \sigma'(x)$.
\end{definition}

\begin{definition}\label{def:subtype}
Given a category $\mathbb{C}$, lists $\alpha, \beta, \gamma \in List\{+,-\}$, functors 
\begin{flalign*}
F : \mathbb{C}^\gamma \rightarrow \mathbb{C}, G : \mathbb{C}^\alpha \rightarrow \mathbb{C}, H : \mathbb{C}^\beta \rightarrow \mathbb{C},
\end{flalign*}s
and a dinatural transformation $\phi : G \rightarrow H$, of type $\cospan{[\alpha]}{\sigma}{[k]}{\tau}{[\beta]}$, the type of $\phi * F$ is given by $\cospan{[\alpha * \gamma]}{\sigma'}{[k * \gamma]}{\tau'}{[\beta * \gamma]}$, where 
\begin{flalign*}
\sigma' &= \bigcup\limits_{i=1}^{|\alpha|} \{(|\gamma| * (i - 1) + j,\ \sigma(i) + j - 1)\ |\ j \in \gamma \},\\
\tau' &= \bigcup\limits_{i=1}^{|\beta|} \{(|\gamma| * (i - 1) + j,\ \tau(i) + j - 1)\ |\ j \in \gamma \}.
\end{flalign*}
\end{definition}

\begin{remark}
By developing a categorical semantics for System $F_\kappa$, the ramification of substitution on naturality types was found. This result is given in Definition \ref{def:subtype}, and is attained by understanding that substitution in System $F_\kappa$, corresponds to the left-whiskering of a dinatural transformation with a functor. By formulating a categorical semantics for System $F_\kappa$, the construction of Petri net models from types can now be formalised.
\end{remark}

\end{document}
