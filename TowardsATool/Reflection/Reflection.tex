% !TeX root = Reflection.tex
\documentclass[../../Reflection.tex]{subfiles}

\begin{document}

\subsection{A brief reflection}
The construction of programs, to solve computational problems, is central to the field of computer science. Although there does not exist any known single solution to every computational problem, an abundance of general problem-solving techniques have been devised. Perhaps the most ubiquitous of these techniques, is the approach of decomposing a problem into smaller pieces. Evidently, this strategy requires a means by which to recompose partial solutions into a single cohesive result.
\par
Intuition is a valuable tool for reformulating complex computational problems as the composition of smaller, manageable sub-problems. Unfortunately, human intuition is not, at present, translatable to a formal, mathematical setting. As a consequence, intuition alone cannot be fashioned into a formal language for describing compositional approaches to solving computational problems. In the pursuit of such a formalisation, it is practical to consider typical composable elements of programs.
\par
Various composable constructs arise across diverse programming paradigms. One such popular example is that of object composition in object-oriented programming languages. A formal theory of object-oriented programming, grounded in mathematical logic, is not commonly recognised, as a consequence of the complexity of mutation and differing subtyping models. The discerning reader might at this point recall that a formal theory of computation, grounded in mathematical logic, has already been discussed in great detail; typed lambda calculi.
\par
Lambda abstractions are a natural choice for composable constructs from which any computation can be described. That is to say, a program can be described as a composition of functions which operate on types. Notably, after defining the constituent functions, this approach avoids any further reference to their underlying implementation. This compositional approach is, in essence, the idea behind the categorical semantics of the simply typed lambda calculus, given in  Section \ref{sec:systemfcat}.
\par
Pragmatically, it is reasonable to wonder whether any practical results can be obtained from such an abstract, category theoretic perspective. Indeed, the most that can be hoped for, with such limited information, is a collection of conditions asserting the equality of specific compositions of functions. As discussed in Section \ref{sec:manyvar}, these are known as coherence conditions and, as a consequence of the Curry-Howard-Lambek correspondence, often translate to non-trivial computational theorems. 
\par
Naturality is amongst the most ubiquitous of coherence conditions, along with the more general conditions of extranaturality and dinaturality. Recall that endofunctors, acting on a cartesian closed category, are used to model polymorphic types in System F by Bainbridge et al. (Section \ref{sec:functorial}). Accordingly, dinatural transformations between these functors correspond to parametrically polymorphic functions. This class of functions share the familiar property of parametricity.
\par
\citeasnoun{ALogicForPP} describe a second-order logic for the terms of System F, and provide a proof that parametricity, from the perspective of Reynold's binary relation semantics, implies dinaturality. As a consequence, Wadler's free theorems can be understood as dinatural coherence conditions. Indeed, Wadler recognised this himself, suggesting that parametricity could be more concisely formulated in terms of lax natural transformations.
\begin{lstlisting}[
  caption={Product of the uncurried church numeral type and the identity},
  label={Haskell-ParametricityFail}
]
applyToFirst :: (a -> a, a, a) -> (a, a)
applyToFirst (f, x, y) = (f x, y)
\end{lstlisting}
Consider the function given in Listing \ref{Haskell-ParametricityFail}. The free theorem for this type would certainly yield a complex equation. Fortunately, there exists an equivalence between free theorems and dinatural coherence conditions. As a consequence, the formulation of coherence conditions as Petri nets, given by McCusker and Santamaria, is applicable to free theorems. This formulation will be given  formally in Section \ref{sec:TypeToPetri}. For the \lstinline{applyToFirst} function, the Petri net corresponding to the free theorem, is given in Figure \ref{fig:applyToFirst}.
\begin{figure}[H]
\begin{center}
\begin{tikzcd}[column sep=0.4em]
  {\tikz[baseline=(char.base)]{\node[shape=circle,draw,minimum size=20pt,inner sep=0pt,fill=lightgray] (char) {};}} &  & {\tikz[baseline=(char.base)]{\node[shape=circle,draw,minimum size=20pt,inner sep=0pt,fill=white] (char) {$f$};}} \arrow[rd] &  & {\tikz[baseline=(char.base)]{\node[shape=circle,draw,minimum size=20pt,inner sep=0pt,fill=white] (char) {$f$};}} \arrow[ld] &  & {\tikz[baseline=(char.base)]{\node[shape=circle,draw,minimum size=20pt,inner sep=0pt,fill=white] (char) {$f$};}} \arrow[llld] \\
   &  &  & {\tikz[baseline=(char.base)]{\node[shape=rectangle,draw,minimum size=12pt,inner sep=0pt,fill=black] (char) {};}} \arrow[lllu] \arrow[ld] \arrow[rd] &  &  &  \\
   &  & {\tikz[baseline=(char.base)]{\node[shape=circle,draw,minimum size=20pt,inner sep=0pt,fill=white] (char) {};}} &  & {\tikz[baseline=(char.base)]{\node[shape=circle,draw,minimum size=20pt,inner sep=0pt,fill=white] (char) {};}} &  & 
  \end{tikzcd}
  =
  \begin{tikzcd}[column sep=0.4em]
    {\tikz[baseline=(char.base)]{\node[shape=circle,draw,minimum size=20pt,inner sep=0pt,fill=lightgray] (char) {$f$};}} &  & {\tikz[baseline=(char.base)]{\node[shape=circle,draw,minimum size=20pt,inner sep=0pt,fill=white] (char) {};}} \arrow[rd] &  & {\tikz[baseline=(char.base)]{\node[shape=circle,draw,minimum size=20pt,inner sep=0pt,fill=white] (char) {};}} \arrow[ld] &  & {\tikz[baseline=(char.base)]{\node[shape=circle,draw,minimum size=20pt,inner sep=0pt,fill=white] (char) {};}} \arrow[llld] \\
     &  &  & {\tikz[baseline=(char.base)]{\node[shape=rectangle,draw,minimum size=12pt,inner sep=0pt,fill=black] (char) {};}} \arrow[lllu] \arrow[ld] \arrow[rd] &  &  &  \\
     &  & {\tikz[baseline=(char.base)]{\node[shape=circle,draw,minimum size=20pt,inner sep=0pt,fill=white] (char) {$f$};}} &  & {\tikz[baseline=(char.base)]{\node[shape=circle,draw,minimum size=20pt,inner sep=0pt,fill=white] (char) {$f$};}} &  & 
    \end{tikzcd}
  \end{center}
  \caption{The free theorem of \lstinline{applyToFirst} as a Petri net.}
\label{fig:applyToFirst}
\end{figure}
Parametricity has the characteristic of equating all occurrences of a type variable. Given only the type of a parametrically polymorphic function, this is a necessary assumption. However, this assumption has an unfortunate consequence; structural information can be lost.
\par
The coherence condition given in Figure \ref{fig:applyToFirst} equates every occurrence of the type variable \lstinline{a} in \lstinline{applyToFirst}. However, by observation of \lstinline{applyToFirst}, it is clear that the last occurences of \lstinline{a} in  the domain and codomain are only equated with each other. The introduction of naturality types, as described in Section \ref{sec:compasreach}, provide a categorical manner in which to describe this condition. This more descriptive condition is depicted as a Petri net in Figure \ref{fig:applyToFirstBetter}.
\begin{figure}[H]
  \begin{center}
  \begin{tikzcd}[column sep=tiny]
    {\tikz[baseline=(char.base)]{\node[shape=circle,draw,minimum size=20pt,inner sep=0pt,fill=lightgray] (char) {};}} & {\tikz[baseline=(char.base)]{\node[shape=circle,draw,minimum size=20pt,inner sep=0pt,fill=white] (char) {$f$};}} \arrow[d] & {\tikz[baseline=(char.base)]{\node[shape=circle,draw,minimum size=20pt,inner sep=0pt,fill=white] (char) {$f$};}} \arrow[ld] & {\tikz[baseline=(char.base)]{\node[shape=circle,draw,minimum size=20pt,inner sep=0pt,fill=white] (char) {$f$};}} \arrow[ld] \\
     & {\tikz[baseline=(char.base)]{\node[shape=rectangle,draw,minimum size=12pt,inner sep=0pt,fill=black] (char) {};}} \arrow[lu] \arrow[d] & {\tikz[baseline=(char.base)]{\node[shape=rectangle,draw,minimum size=12pt,inner sep=0pt,fill=black] (char) {};}} \arrow[d] &  \\
     & {\tikz[baseline=(char.base)]{\node[shape=circle,draw,minimum size=20pt,inner sep=0pt,fill=white] (char) {};}} & {\tikz[baseline=(char.base)]{\node[shape=circle,draw,minimum size=20pt,inner sep=0pt,fill=white] (char) {};}} & 
    \end{tikzcd}
    =
    \begin{tikzcd}[column sep=tiny]
      {\tikz[baseline=(char.base)]{\node[shape=circle,draw,minimum size=20pt,inner sep=0pt,fill=lightgray] (char) {$f$};}} & {\tikz[baseline=(char.base)]{\node[shape=circle,draw,minimum size=20pt,inner sep=0pt,fill=white] (char) {};}} \arrow[d] & {\tikz[baseline=(char.base)]{\node[shape=circle,draw,minimum size=20pt,inner sep=0pt,fill=white] (char) {};}} \arrow[ld] & {\tikz[baseline=(char.base)]{\node[shape=circle,draw,minimum size=20pt,inner sep=0pt,fill=white] (char) {};}} \arrow[ld] \\
       & {\tikz[baseline=(char.base)]{\node[shape=rectangle,draw,minimum size=12pt,inner sep=0pt,fill=black] (char) {};}} \arrow[lu] \arrow[d] & {\tikz[baseline=(char.base)]{\node[shape=rectangle,draw,minimum size=12pt,inner sep=0pt,fill=black] (char) {};}} \arrow[d] &  \\
       & {\tikz[baseline=(char.base)]{\node[shape=circle,draw,minimum size=20pt,inner sep=0pt,fill=white] (char) {$f$};}} & {\tikz[baseline=(char.base)]{\node[shape=circle,draw,minimum size=20pt,inner sep=0pt,fill=white] (char) {$f$};}} & 
      \end{tikzcd}
    \end{center}
\caption{A more descriptive condition on \lstinline{applyToFirst}.}
\label{fig:applyToFirstBetter}
\end{figure}

Furthermore, by observing the implementation of \lstinline{applyToFirst}, it is reasonable to conclude that its provided type is unnecessarily specialised. Indeed, given only the function body of \lstinline{applyToFirst}, a type inference algorithm would find its type to be \lstinline{(a -> b, a, c)} \lstinline{-> (b, c)}. This type contains three unique type variables, each of which corresponds to a different naturality condition. More precisely, the type is said to be natural in \lstinline{b} and \lstinline{c}, and extranatural in \lstinline{a}. The Petri nets corresponding to these naturality conditions are given in Figure \ref{fig:applyToFirstLast}. Read from left to right, these conditions are in the order \lstinline{a}, \lstinline{b}, \lstinline{c}.

\begin{figure}[H]
  \begin{center}
    \begin{tikzcd}
    {\tikz[baseline=(char.base)]{\node[shape=circle,draw,minimum size=20pt,inner sep=0pt,fill=lightgray] (char) {};}} &  & {\tikz[baseline=(char.base)]{\node[shape=circle,draw,minimum size=20pt,inner sep=0pt,fill=white] (char) {$f$};}} \arrow[ld] \\
      & {\tikz[baseline=(char.base)]{\node[shape=rectangle,draw,minimum size=12pt,inner sep=0pt,fill=black] (char) {};}} \arrow[lu] & 
    \end{tikzcd}
    \qquad\qquad
    \begin{tikzcd}
    {\tikz[baseline=(char.base)]{\node[shape=circle,draw,minimum size=20pt,inner sep=0pt,fill=white] (char) {$f$};}} \arrow[d] \\
    {\tikz[baseline=(char.base)]{\node[shape=rectangle,draw,minimum size=12pt,inner sep=0pt,fill=black] (char) {};}} \arrow[d] \\
    {\tikz[baseline=(char.base)]{\node[shape=circle,draw,minimum size=20pt,inner sep=0pt,fill=white] (char) {};}}
    \end{tikzcd}
    \qquad\qquad
    \begin{tikzcd}
    {\tikz[baseline=(char.base)]{\node[shape=circle,draw,minimum size=20pt,inner sep=0pt,fill=white] (char) {$f$};}} \arrow[d] \\
    {\tikz[baseline=(char.base)]{\node[shape=rectangle,draw,minimum size=12pt,inner sep=0pt,fill=black] (char) {};}} \arrow[d] \\
    {\tikz[baseline=(char.base)]{\node[shape=circle,draw,minimum size=20pt,inner sep=0pt,fill=white] (char) {};}}
    \end{tikzcd}
  \end{center}
\caption{Naturality condition of \lstinline{(a -> b, a, c)} \lstinline{-> (b, c)}.}
\label{fig:applyToFirstLast}
\end{figure}

Two clear benefits, relating to the graphical expression of free theorems, have emerged; brevity and generality. The described Petri nets allow for the depiction of the usual free theorems, and with the addition of naturality types, are also able to depict more descriptive conditions. 
\par
Furthermore, when the domain and codomain of two Petri net models have an identical structure, they may be composed simply by pasting the respective domain along the codomain. Such a simple operation is unfortunately not always possible, even if the underlying types are composable. This is a consequence of the difference between composition in a Hindley-Milner type system, and composition of dinatural transformations. The described problem will be returned to shortly.
\par
In summary, a graphical representation of dinatural coherence conditions and, equivalently, free theorems on parametrically polymorphics types, has been given. This representation provides  a concise, general and composable way to explore the relationship between parametricity and dinaturality. Thus, a computational tool is proposed for generating Petri nets from parametrically polymorphic types, which are operational and composable. 
\par
To achieve the proposed aim of a computational tool, it is necessary to determine a formal system of types on which to operate. This system should be suitable for expressing dinatural transformations in the generalised functorial calculus described by McCusker and Santamaria. As previously described, it will also be necessary to unify the ideas of composition in the chosen type system and of the underlying dinatural transformations. Finally, a formal method for generating Petri nets from types in the chosen system, will be given. As specified, the first task is to designate an appropriate type system.

\end{document}
