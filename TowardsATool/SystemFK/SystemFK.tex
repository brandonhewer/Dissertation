% !TeX root = SystemFK.tex
\documentclass[../../Dissertation.tex]{subfiles}

\begin{document}

\subsection{System $F_\kappa$}
A novel extension to the Hindley-Milner system is proposed, termed System $F_\kappa$. System $F_\kappa$ extends the Hindley-Milner type system with a restricted form of kinding. Higher-kinded types, in System $F_\kappa$, are isomorphic to the types of multifunctors acting on the cartesian closed category of Hindley-Milner types. Instantiations of higher-kinded types are known as type constructors, and are required to be functorial in System $F_\kappa$.
\par
Higher-kinded types in System $F_\kappa$ are of the form
\begin{equation}\label{eq:hkt}
  K ::= *\ |\ * \Rightarrow K\ |\ *^- \Rightarrow K.
\end{equation}
The syntax of higher-kinded types given in (\ref{eq:hkt}) has two noteworthy departures from that of System $F_\omega$. In constrast to System $F_\omega$, two primitive kinds are given, $*$ and $*^-$, which are used to denote positive and negative variance, respectively. In addition, the syntax of kinds in System $F_\kappa$ prevents the construction of higher-order kinds. These departures are necessary in establishing the requirement that type constructors in System $F_\kappa$ satisfy the functorial laws.
\par
To construct a computational tool, which utilises the types of System $F_\kappa$, it is required that the outlined ideas be made precise. An extension of the unification algorithm to incorporate type constructors will also be essential for composing types in System $F_\kappa$. As  such, a formalisation of System $F_\kappa$ is given.

\begin{notation}
  Given a set $X$, $List\ X$ will be used to denote the free monoid on $X$. An element of $List\ X$ is a finite list of objects which exist in $X$.
\end{notation}

\begin{definition}
Given a list $\theta \in List\{+, -\}$, $n = |\theta|$, a kind $\upsilon^\theta$, in System $F_\kappa$, is of the form 
\begin{flalign*}
*^{\theta_1} \Rightarrow *^{\theta_2} \Rightarrow \ldots \Rightarrow *^{\theta_n} \Rightarrow *,
\end{flalign*}
where $*^+ = *$.
\end{definition}

\begin{definition} 
The well-formedness judgements for kinding in System $F_\kappa$ are given by
\begin{equation*}
  \begin{prooftree}
    \hypo{T : \upsilon^\theta \in \Delta}
    \infer1[KVar]{\Delta \vdash T : \upsilon^\theta}
  \end{prooftree}
\end{equation*}
\begin{equation*}
  \begin{prooftree}
    \hypo{\Delta \vdash X : *}
    \hypo{\Delta \vdash Y : *}
    \infer2[KArr]{\Delta \vdash X \rightarrow Y : *}
  \end{prooftree}
\end{equation*}
\begin{equation*}
  \begin{prooftree}
    \hypo{\Delta,\ X : * \vdash T : \upsilon^\theta}
    \infer1[KAbs]{\Delta \vdash \lambda X.T : * \Rightarrow \upsilon^\theta}
  \end{prooftree}\qquad
  \begin{prooftree}
      \hypo{\Delta,\ X : *^- \vdash T : \upsilon^\theta}
      \infer1[KAbs-]{\Delta \vdash \lambda X.T : *^- \Rightarrow \upsilon^\theta}
  \end{prooftree}
\end{equation*}
\begin{equation*}
  \begin{prooftree}
    \hypo{\Delta \vdash T : * \rightarrow \upsilon^\theta}
    \hypo{\Delta \vdash X : *}
    \infer2[KApp]{\Delta \vdash T X : \upsilon^\theta}
  \end{prooftree}
\end{equation*}
\begin{equation*} 
  \begin{prooftree}
    \hypo{\Delta \vdash T : *^- \rightarrow \upsilon^\theta}
    \hypo{\Delta \vdash X : *}
    \infer2[KApp-]{\Delta \vdash T X : \upsilon^\theta}
  \end{prooftree}
\end{equation*}
\begin{equation*}
  \begin{prooftree}
    \hypo{\Delta \vdash X : *}
    \infer1[KGen]{\Delta \vdash \forall \left(F : \upsilon^\theta\right).X : *}
  \end{prooftree}
\end{equation*}
\end{definition}

\begin{definition}
Types in System $F_\kappa$ are of the form
\begin{flalign*}
c\ &|\ \alpha\ |\ \tau_1 \times \tau_2\ |\ \tau_1 \rightarrow \tau_2\ |\ \forall \alpha. \sigma\\
&|\ T\ \tau_1\ \tau_2\ \ldots\ \tau_n\ |\ F\ \tau_1\ \tau_2\ \ldots\ \tau_n\ |\ \forall (F : \upsilon^\theta). \sigma,
\end{flalign*}
where $c$ is a primitive type, $\alpha$ is a type variable, $T$ is a type constructor and $F$ is a polymorphic type constructor.
\end{definition}

\begin{definition}
Given a kinding context $\Delta$, the well-formedness judgements for typing in System $F_\kappa$ are given by
\begin{equation*}
  \begin{prooftree}
    \hypo{x : \sigma \in \Gamma}
    \hypo{\Delta \vdash \Gamma}
    \infer2[TVar]{\Delta,\ \Gamma \vdash x : \sigma}
  \end{prooftree}
\end{equation*}
\begin{equation*}
  \begin{prooftree}
    \hypo{\Delta \vdash \sigma : *}
    \hypo{\Delta,\ \Gamma \vdash e : \sigma'}
    \hypo{\sigma' \sqsubseteq \sigma}
    \infer3[TInst]{\Delta,\ \Gamma \vdash e : \sigma}
  \end{prooftree}
\end{equation*}
\begin{equation*}
  \begin{prooftree}
    \hypo{\Delta \vdash \tau : *}
    \hypo{\Delta,\ \Gamma,\ x : \tau \vdash e : \tau'}
    \infer2[TAbs]{\Delta,\ \Gamma \vdash \lambda x.e : \tau \rightarrow \tau'}
  \end{prooftree}
\end{equation*}
\begin{equation*}
  \begin{prooftree}
    \hypo{\Delta,\ \Gamma \vdash e : \tau \rightarrow \tau'}
    \hypo{\Delta,\ \Gamma \vdash e' : \tau}
    \infer2[TApp]{\Gamma \vdash e\ e' : \tau'}
  \end{prooftree}
\end{equation*}
\begin{equation*} 
  \begin{prooftree}
    \hypo{\Delta,\ \Gamma \vdash e : \sigma}
    \hypo{\Delta,\ \Gamma,\ x : \sigma \vdash e' : \tau}
    \infer2[TLet]{\Delta,\ \Gamma \vdash \textbf{let } x = e \textbf{ in } e' : \tau}
  \end{prooftree}
\end{equation*}
\begin{equation*}
  \begin{prooftree}
    \hypo{\Delta,\ \Gamma \vdash e : \sigma}
    \hypo{\alpha \notin \textit{free}(\Gamma)}
    \infer2[TGen]{\Delta,\ \Gamma \vdash e : \forall \alpha.\sigma}
  \end{prooftree}
\end{equation*}
\begin{equation*}
  \begin{prooftree}
    \hypo{\Delta,\ \Gamma \vdash e : \tau}
    \hypo{\Delta,\ \Gamma \vdash e' : \tau'}
    \infer2[TProd]{\Delta,\ \Gamma \vdash \langle e, e' \rangle : \tau \times \tau'}
  \end{prooftree}
\end{equation*}
\begin{equation*}
  \begin{prooftree}
    \hypo{\Delta,\ \Gamma \vdash e : \tau \times \tau'}
    \infer1[TProj]{\Delta,\ \Gamma \vdash \pi_i(e) : \tau(i)}
  \end{prooftree}\qquad
  \begin{prooftree}
    \infer0[TUnit]{\Delta,\ \Gamma \vdash \langle \rangle : 1}
  \end{prooftree}
\end{equation*}
\begin{equation*}
  \begin{prooftree}
    \hypo{\Delta,\ \Gamma \vdash e : \sigma}
    \hypo{F \notin \textit{free}(\Gamma)}
    \infer2[TKGen]{\Delta,\ \Gamma \vdash e : \forall (F : \upsilon^\theta).\sigma}
  \end{prooftree}
\end{equation*}
\begin{equation*}
  \begin{prooftree}
    \hypo{\Delta,\ \Gamma \vdash e : \forall (F : \upsilon^\theta).\sigma}
    \hypo{\Delta \vdash T : \upsilon^\theta}
    \infer2[TKApp]{\Delta,\ \Gamma \vdash e : \sigma[T/F]}
  \end{prooftree}
\end{equation*}
\end{definition}

\begin{definition}
Let $\theta \in List\{+, -\}$ be a list, where $n = |\theta|$. Define $\upsilon^\theta$ to be the kind of the type constructor $T$, and polymorphic  type constructor $F$. The function $\textit{free}$, which computes the set of free variables in a term, is extended in System $F_\kappa$ by
\begin{flalign*}
  \textit{free}(T\ \tau_1\ \tau_2\ \ldots\ \tau_n) &= \bigcup\limits_{i=1}^{n} \textit{free}(\tau_i)\\
  \textit{free}(F\ \tau_1\ \tau_2\ \ldots\ \tau_n) &= \{ F \} \cup \bigcup\limits_{i=1}^{n} \textit{free}(\tau_i)\\
  \textit{free}(\forall (F : \upsilon^\theta).\tau) &= \{ x \in \textit{free}(\tau) : x \neq F \}.
\end{flalign*}
\end{definition}

\begin{example} Let $F : * \Rightarrow * \Rightarrow *$ and $G : * \Rightarrow *$ be polymorphic type constructors in System $F_\kappa$. Consider the type
\begin{equation*}
  \tau = \forall F.\ \forall \alpha. F\ (G\ \alpha)\ \beta.
\end{equation*}
The free variables in $\tau$ are given by
\begin{equation*}
  \textit{free}(\tau) = \{G,\ \beta\}.
\end{equation*}
\end{example}

\begin{definition}
Let $\theta \in List\{+, -\}$ be a list, where $n = |\theta|$.  Define $\upsilon^\theta$ to be the kind of the type constructor $T$ and polymorphic type constructors $F$, $G$, in System $F_\kappa$. The Hindley-Milner substitution rules are extended for System $F_\kappa$ as follows
\begin{flalign*}
  ((\forall H : \upsilon^\theta).\tau)[T/G] &=
  \begin{cases}
    (\forall H : \upsilon^\theta).\tau & \text{if } H = G\\
    (\forall J : \upsilon^\theta).(\tau[J/H][T/G]) & \text{otherwise}
  \end{cases},\\
  & \qquad \text{where } J \notin \textit{free}(\tau)\\
  (F\ \tau_1\ \tau_2\ \ldots\ \tau_n)[T/G] &= 
  \begin{cases}
    T\ \tau_1[T/G]\ \tau_2[T/G]\ \ldots\ \tau_n[T/G] & \text{if } F = G\\
    F\ \tau_1[T/G]\ \tau_2[T/G]\ \ldots\ \tau_n[T/G] & \text{otherwise}
  \end{cases},\\
  (F\ \tau_1\ \tau_2\ \ldots\ \tau_n)[\sigma/\alpha] &= F\ \tau_1[\sigma/\alpha]\ \tau_2[\sigma/\alpha]\ \ldots\ \tau_n[\sigma/\alpha],\\
  (T\ \tau_1\ \tau_2\ \ldots\ \tau_n)[\sigma/\alpha] &= T\ \tau_1[\sigma/\alpha]\ \tau_2[\sigma/\alpha]\ \ldots\ \tau_n[\sigma/\alpha].
\end{flalign*}
\end{definition}

\begin{example}
  Let $F : * \Rightarrow * \Rightarrow *$ and $G : * \Rightarrow *$ be polymorphic type constructors in System $F_\kappa$. Consider the type
  \begin{equation*}
    \tau = \forall F.\ \forall \alpha.F\ (G\ \alpha)\ (F\ \alpha\ \beta).
  \end{equation*}
  Substitution of the type variable $\beta$, produces
  \begin{equation*}
    \tau[\sigma/\beta] = \forall F.\ \forall \gamma.F\ (G\ \gamma)\ (F\ \gamma\ \sigma),
  \end{equation*}
  while substitution of the type constructor variable $G$, produces
  \begin{equation*}
    \tau[(T : * \Rightarrow *)/G] = \forall H.\ \forall \alpha.H\ (T\ \alpha)\ (H\ \alpha\ \sigma).
  \end{equation*}
\end{example}

\begin{definition}
Let $\theta \in List\{+, -\}$ be a list, where $n = |\theta|$, and $\upsilon^\theta$ be a kind in System $F_\kappa$. Given a type constructor $T : \upsilon^\theta$ there must exist a polymorphic function
\begin{flalign*}
  \mathcal{F}_T : (\alpha_1 \rightarrow \alpha_1') \rightarrow \ldots \rightarrow (\alpha_n \rightarrow \alpha_n') &\rightarrow T(\beta_1,\ldots,\beta_n) \rightarrow T(\beta_1',\ldots,\beta_n'),\\
  \text{where } \beta_i =
  \begin{cases}
    \alpha_i & \text{if } \theta_i = +\\
    \alpha_i' & \text{otherwise}
  \end{cases}&,\ 
  \beta_i' =
  \begin{cases}
    \alpha_i' & \text{if } \theta_i = +\\
    \alpha_i & \text{otherwise}
  \end{cases},
\end{flalign*}
such that given the identity $\iota_\tau = \lambda x^\tau.x$,
\begin{flalign*}
  \mathcal{F}_T(\iota_{\tau_1},\ \iota_{\tau_2},\ \ldots,\ \iota_{\tau_n}) &= \iota_{T^(\tau_1,\ \tau_2,\ \ldots,\ \tau_n)},
\end{flalign*}
and given terms
\begin{flalign*}
  f_1 : \tau_1 \rightarrow \tau_1',\ f_2 : \tau_2 \rightarrow \tau_2',\ \ldots,\ f_n : \tau_n \rightarrow \tau_n',\\
  g_1 : \tau_1' \rightarrow \sigma_1,\ g_2 : \tau_2' \rightarrow \sigma_2,\ \ldots,\ g_n : \tau_n' \rightarrow \sigma_n,
\end{flalign*}
the following equality holds:
\begin{gather*}
  \mathcal{F}_T(g_1 \circ f_1,\ g_2 \circ f_2,\ \ldots,\ g_n \circ f_n) = \mathcal{F}_T(h_1,\ h_2,\ \ldots,\ h_n) \circ \mathcal{F}(h_1',\ h_2',\ \ldots,\ h_n'),\\
  \text{where } h_i =
  \begin{cases}
    g_i & \text{if } \theta_i = +\\
    f_i & \text{otherwise}
  \end{cases},\ 
  h_i' =
  \begin{cases}
    f_i & \text{if } \theta_i = +\\
    g_i & \text{otherwise}
  \end{cases}.
\end{gather*}
Note that this is an assertion that every type constructor in System $F_\kappa$ must be functorial.
\end{definition}

\begin{definition}
Given a list $\theta \in List\{+,-\}$, define $\theta^{+} = \theta$ and $\theta^{-}$ to be the list in which the sign of every element is flipped (i.e. every $+$ is switched to $-$ and vice versa).
\end{definition}

\begin{definition}
Given a list $\theta \in List\{+,-\}$, $\theta$ is said to `match' a pattern $\gamma \in List\{+,-\}$, $n = |\gamma|$, with `substitution' $s \in List\{+,-\}$, denoted $\theta = \gamma(s)$, if
\begin{flalign*}
\theta = s^{\gamma_1} \mdoubleplus s^{\gamma_2} \mdoubleplus ... \mdoubleplus s^{\gamma_n}
\end{flalign*}
where $\mdoubleplus$ is the concatenation operator. Note that $n|s| = |\theta|$.
\end{definition}

\begin{example}
  The list 
  \begin{center}
  \begin{tabular}{c}
  \begin{lstlisting}
    [+, -, -, +],
  \end{lstlisting}
  \end{tabular}
  \end{center}
  matches the pattern
  \begin{center}
  \begin{tabular}{c}
  \begin{lstlisting}
    [-, +],
  \end{lstlisting}
  \end{tabular}
  \end{center}
  with substitution
  \begin{center}
  \begin{tabular}{c}
  \begin{lstlisting}
    [-, +].
  \end{lstlisting}
  \end{tabular}
  \end{center}
  This can intuitively be understood by substituting \lstinline{[-, +]}, whenever \lstinline{+} appears in \lstinline{[-, +]}, and substituting the list with flipped signs, \lstinline{[+, -]}, whenever \lstinline{-} appears. This operation would yield the nested list
  \begin{center}
  \begin{tabular}{c}
  \begin{lstlisting}
    [[+, -], [-, +]],
  \end{lstlisting}
  \end{tabular}
  \end{center}
  which can be flattened to produce the expected result of \lstinline{[+, -, -, +]}.
\end{example}

\begin{definition}\label{def:unify}
Given the Hindley-Milner unification function $U$, define the function $U^*$ as
\begin{flalign*}
  U^*([\ ], [\ ], S) &= S,\\
  U^*([\tau_1,\tau_2,\ldots,\tau_n],[\sigma_1,\sigma_2,\ldots,\sigma_n], S) &= X \circ U^*([\tau_2,\ldots,\tau_n],[\sigma_2,\ldots,\sigma_n], Y)\\
  &\quad\ \text{where } X = U(\tau_1[S], \sigma_1[S])\\
  &\qquad \qquad \ Y = S \circ X.
\end{flalign*}
Let $\theta$, $\gamma \in List\{+, -\}$ be lists, where $n = |\theta|$, $m = |\gamma|$, and $\upsilon^\theta$ and $\omega^\gamma$ be kinds in System $F_\kappa$. Denote 
\begin{flalign*}
  [\![\tau]\!]_n = [\tau_1,\tau_2,\ldots,\tau_n],\\
  [\![\tau]\!]_{i:n} = [\tau_i,\tau_{i+1},\ldots,\tau_n].
\end{flalign*}
Given type constructors $T_1 : \upsilon^\theta$, $T_2 : \omega^\gamma$, and polymorphic type constructors $F : \upsilon^\theta$, $G : \omega^\gamma$, $U$ can be extended in System $F_\kappa$ as follows
\begin{flalign*}
  U (\tau_1 \times \tau_2, \sigma_1 \times \sigma_2) &= U^*([\![\tau]\!]_2, [\![\sigma]\!]_2, [\ ]),\\
  U(T_1\ \tau_1\ \ldots\ \tau_n,\ T_2\ \sigma_1\ \ldots\ \sigma_m) &=
  \begin{cases}
    U^*([\![\tau]\!]_n, [\![\sigma]\!]_n, [\ ]) & \text{if } T_1 = T_2\\
    \bot & \text{otherwise}
  \end{cases},\\\\
  U(F\ \tau_1\ \ldots\ \tau_n,\ G\ \sigma_1\ \ldots\ \sigma_m) &=
  \begin{cases}
    Y & \text{if } \exists s : \theta = \gamma(s) \\
    \bot & \text{otherwise}
  \end{cases},\\
  &\qquad \text{where } Y = X_1 \circ X_2 \circ \ldots \circ X_{|s|} \circ [F/G]\\
  &\qquad \qquad \ \: X_i = U^*([\![\tau]\!]_{1 + (i - 1)m : im}, [\![\sigma]\!]_m, [\ ]),\\\\
  U(T_1\ \tau_1\ \ldots\ \tau_n,\ F\ \sigma_1\ \ldots\ \sigma_n) &=
  U^*([\![\tau]\!]_n, [\![\sigma]\!]_n, [T_1/F]),\\
  U(F\ \tau_1\ \ldots\ \tau_n,\ T_1\ \sigma_1\ \ldots\ \sigma_n) &= 
  U^*([\![\tau]\!]_n, [\![\sigma]\!]_n, [T_1/F]).
\end{flalign*}
\end{definition}

\begin{example}
  Let $F : *^- \Rightarrow * \Rightarrow *$ and $G : * \Rightarrow *^- \Rightarrow *^- \Rightarrow * \Rightarrow *$ be polymorphic type constructors. Consider the types
  \begin{flalign*}
    \tau &= \forall F.\ \forall \alpha.F\ (\alpha \rightarrow \alpha)\ (\alpha \rightarrow \alpha),\\
    \sigma &= \forall G.\ \forall \beta.\ \forall \gamma.G\ \beta\ \beta\ \beta\ (\gamma \rightarrow \gamma).
  \end{flalign*}
  There exists a unification of $\tau$ and $\sigma$, given by
  \begin{equation*}
    S = U(\tau,\ \sigma) = [(\alpha \rightarrow \alpha)/\beta] \circ [\alpha/\gamma] \circ [G/F],
  \end{equation*}
  such that
  \begin{equation*}
    \tau[S] = \sigma[S] = G\ (\alpha \rightarrow \alpha)\ (\alpha \rightarrow \alpha)\ (\alpha \rightarrow \alpha)\ (\alpha \rightarrow \alpha). 
  \end{equation*}
\end{example}

\begin{remark}
Although the given description of System $F_\kappa$ provides the necessary framework for the desired computation tool, it is far from comprehensive. An exploration of additional properties of System $F_\kappa$ is left to future work.
\par
System $F_\kappa$ describes a  generalised functorial calculus over the types of the Hindley-Milner type system. Accordingly, further investigation of System $F_\kappa$ would benefit from a description of the categorical semantics which underlies the involved functors and dinatural transformations.
\end{remark}

\end{document}
