% !TeX root = SystemFK.tex
\documentclass[../../Dissertation.tex]{subfiles}

\begin{document}

\subsection{System $F_\kappa$}
A novel extension to the Hindley-Milner system is proposed, termed System $F_\kappa$. System $F_\kappa$ extends the Hindley-Milner type system with a restricted form of kinding. Higher-kinded types, in System $F_\kappa$, are isomorphic to the types of multifunctors acting on the cartesian closed category of Hindley-Milner types. Instantiations of higher-kinded types are known as type constructors, and are required to be functorial in System $F_\kappa$.
\par
Higher-kinded types in System $F_\kappa$ are of the form
\begin{equation}\label{eq:hkt}
  K ::= *\ |\ * \Rightarrow K\ |\ *^- \Rightarrow K.
\end{equation}
The syntax of higher-kinded types given in (\ref{eq:hkt}) has two noteworthy departures from that of System $F_\omega$. In constrast to System $F_\omega$, two primitive kinds are given, $*$ and $*^-$, which are used to denote positive and negative variance, respectively. In addition, the syntax of kinds in System $F_\kappa$ prevents the construction of higher-order kinds. These departures are necessary in establishing the requirement that type constructors in System $F_\kappa$ satisfy the functorial laws.
\par
To construct a computational tool, which utilises the types of System $F_\kappa$, it is required that the outlined ideas be made precise. An extension of the unification algorithm to incorporate type constructors will also be essential for composing types in System $F_\kappa$. As  such, a formalisation of System $F_\kappa$ is given.
\\\\
{
\noindent
\textbf{Definition 1.} Given a list $\theta \in List\{+, -\}$, $n = |\theta|$, a higher-kinded type $\upsilon^\theta$ in System $F_\kappa$ is of the form 
\begin{flalign*}
*^{\theta_1} \Rightarrow *^{\theta_2} \Rightarrow \ldots \Rightarrow *^{\theta_n} \Rightarrow *,
\end{flalign*}
where $*^+ = *$.\\
}

{
\noindent
\textbf{Definition 2.} The well-formedness judgements for kinding in System $F_\kappa$ are given by
\begin{equation*}
  \begin{prooftree}
    \hypo{T : \upsilon^\theta \in \Delta}
    \infer1[KVar]{\Delta \vdash T : \upsilon^\theta}
  \end{prooftree}
\end{equation*}
\begin{equation*}
  \begin{prooftree}
    \hypo{\Delta \vdash X : *}
    \hypo{\Delta \vdash Y : *}
    \infer2[KArr]{\Delta \vdash X \rightarrow Y : *}
  \end{prooftree}
\end{equation*}
\begin{equation*}
  \begin{prooftree}
    \hypo{\Delta,\ X : * \vdash T : \upsilon^\theta}
    \infer1[KAbs]{\Delta \vdash \lambda X.T : * \Rightarrow \upsilon^\theta}
  \end{prooftree}\qquad
  \begin{prooftree}
      \hypo{\Delta,\ X : *^- \vdash T : \upsilon^\theta}
      \infer1[KAbs-]{\Delta \vdash \lambda X.T : *^- \Rightarrow \upsilon^\theta}
  \end{prooftree}
\end{equation*}
\begin{equation*}
  \begin{prooftree}
    \hypo{\Delta \vdash T : * \rightarrow \upsilon^\theta}
    \hypo{\Delta \vdash X : *}
    \infer2[KApp]{\Delta \vdash T X : \upsilon^\theta}
  \end{prooftree}
\end{equation*}
\begin{equation*} 
  \begin{prooftree}
    \hypo{\Delta \vdash T : *^- \rightarrow \upsilon^\theta}
    \hypo{\Delta \vdash X : *}
    \infer2[KApp-]{\Delta \vdash T X : \upsilon^\theta}
  \end{prooftree}
\end{equation*}
\begin{equation*}
  \begin{prooftree}
    \hypo{\Delta \vdash X : *}
    \infer1[KGen]{\Delta \vdash \forall \left(F : \upsilon^\theta\right).X : *}
  \end{prooftree}
\end{equation*}
}

{
\noindent
\textbf{Definition 3.} Given a kinding context $\Delta$, the well-formedness judgements for typing in System $F_\kappa$ are given by
\begin{equation*}
  \begin{prooftree}
    \hypo{x : \sigma \in \Gamma}
    \hypo{\Delta \vdash \Gamma}
    \infer2[TVar]{\Delta,\ \Gamma \vdash x : \sigma}
  \end{prooftree}
\end{equation*}
\begin{equation*}
  \begin{prooftree}
    \hypo{\Delta \vdash \sigma : *}
    \hypo{\Delta,\ \Gamma \vdash e : \sigma'}
    \hypo{\sigma' \sqsubseteq \sigma}
    \infer3[TInst]{\Delta,\ \Gamma \vdash e : \sigma}
  \end{prooftree}
\end{equation*}
\begin{equation*}
  \begin{prooftree}
    \hypo{\Delta \vdash \tau : *}
    \hypo{\Delta,\ \Gamma,\ x : \tau \vdash e : \tau'}
    \infer2[TAbs]{\Delta,\ \Gamma \vdash \lambda x.e : \tau \rightarrow \tau'}
  \end{prooftree}
\end{equation*}
\begin{equation*}
  \begin{prooftree}
    \hypo{\Delta,\ \Gamma \vdash e : \tau \rightarrow \tau'}
    \hypo{\Delta,\ \Gamma \vdash e' : \tau}
    \infer2[TApp]{\Gamma \vdash e\ e' : \tau'}
  \end{prooftree}
\end{equation*}
\begin{equation*} 
  \begin{prooftree}
    \hypo{\Delta,\ \Gamma \vdash e : \sigma}
    \hypo{\Delta,\ \Gamma,\ x : \sigma \vdash e' : \tau}
    \infer2[TLet]{\Delta,\ \Gamma \vdash \textbf{let } x = e \textbf{ in } e' : \tau}
  \end{prooftree}
\end{equation*}
\begin{equation*}
  \begin{prooftree}
    \hypo{\Delta,\ \Gamma \vdash e : \sigma}
    \hypo{\alpha \notin \textit{free}(\Gamma)}
    \infer2[TGen]{\Delta,\ \Gamma \vdash e : \forall \alpha.\sigma}
  \end{prooftree}
\end{equation*}
\begin{equation*}
  \begin{prooftree}
    \hypo{\Delta,\ \Gamma \vdash e : \sigma}
    \hypo{F \notin \textit{free}(\Gamma)}
    \infer2[TKGen]{\Delta,\ \Gamma \vdash e : \forall (F : \upsilon^\theta).\sigma}
  \end{prooftree}
\end{equation*}
\begin{equation*}
  \begin{prooftree}
    \hypo{\Delta,\ \Gamma \vdash e : \forall (F : \upsilon^\theta).\sigma}
    \hypo{\Delta \vdash T : \upsilon^\theta}
    \infer2[TKApp]{\Delta,\ \Gamma \vdash e : \sigma[T/F]}
  \end{prooftree}
\end{equation*}
}

{
\noindent
\textbf{Definition 4.} Let $\theta \in List\{+, -\}$ be a list, where $n = |\theta|$, and $\upsilon^\theta$ be a higher-kinded type, in System $F_\kappa$. The function $\textit{free}$, which computes the set of free variables in a term, is extended in System $F_\kappa$ by
\begin{flalign*}
  \textit{free}((T : \upsilon^\theta)\ \tau_1\ \tau_2\ \ldots\ \tau_n) &= \{ T \} \cup \bigcup\limits_{i=1}^{n} \textit{free}(\tau_i)\\
  \textit{free}(\forall (F : \upsilon^\theta).\tau) &= \{ x \in \textit{free}(\tau) : x \neq F \}.
\end{flalign*}
}

{
\noindent
\textbf{Definition 5.} Let $\theta \in List\{+, -\}$ be a list, where $n = |\theta|$, and $\upsilon^\theta$ be a higher-kinded type, in System $F_\kappa$. Given a type constructor $T : \upsilon^\theta$ and polymorphic type constructors $F : \upsilon^\theta$, $G : \upsilon^\theta$, the Hindley-Milner substitution rules are extended for System $F_\kappa$ as follows
\begin{flalign*}
  ((\forall H : \upsilon^\theta).\tau)[T/G] &=
  \begin{cases}
    (\forall H : \upsilon^\theta).\tau & \text{if } H = G\\
    (\forall J : \upsilon^\theta).(\tau[J/H][T/G]) & \text{otherwise}
  \end{cases},\\
  & \qquad \text{where } J \notin \textit{free}(\tau)\\
  (F\ \tau_1\ \tau_2\ \ldots\ \tau_n)[T/G] &= 
  \begin{cases}
    T\ \tau_1[T/G]\ \tau_2[T/G]\ \ldots\ \tau_n[T/G] & \text{if } F = G\\
    F\ \tau_1[T/G]\ \tau_2[T/G]\ \ldots\ \tau_n[T/G] & \text{otherwise}
  \end{cases},\\
  (F\ \tau_1\ \tau_2\ \ldots\ \tau_n)[\sigma/\alpha] &= F\ \tau_1[\sigma/\alpha]\ \tau_2[\sigma/\alpha]\ \ldots\ \tau_n[\sigma/\alpha],\\
  (T\ \tau_1\ \tau_2\ \ldots\ \tau_n)[\sigma/\alpha] &= T\ \tau_1[\sigma/\alpha]\ \tau_2[\sigma/\alpha]\ \ldots\ \tau_n[\sigma/\alpha].
\end{flalign*}
}

{
\noindent
\textbf{Definition 6.} Let $\theta \in List\{+, -\}$ be a list, where $n = |\theta|$, and $\upsilon^\theta$ be a higher-kinded type, in System $F_\kappa$. Given a type constructor $T : \upsilon^\theta$ there must exist a polymorphic function
\begin{flalign*}
  \mathcal{F} : (\alpha_1 \rightarrow \alpha_1') \rightarrow \ldots \rightarrow (\alpha_n \rightarrow \alpha_n') &\rightarrow T(\beta_1,\ldots,\beta_n) \rightarrow T(\beta_1',\ldots,\beta_n'),\\
  \text{where } \beta_i =
  \begin{cases}
    \alpha_i & \text{if } \theta_i = +\\
    \alpha_i' & \text{otherwise}
  \end{cases}&,\ 
  \beta_i' =
  \begin{cases}
    \alpha_i' & \text{if } \theta_i = +\\
    \alpha_i & \text{otherwise}
  \end{cases},
\end{flalign*}
such that given the identity $\iota_\tau = \lambda x^\tau.x$,
\begin{flalign*}
  \mathcal{F}(\iota_{\tau_1},\ \iota_{\tau_2},\ \ldots,\ \iota_{\tau_n}) &= \iota_{T^(\tau_1,\ \tau_2,\ \ldots,\ \tau_n)},
\end{flalign*}
and given terms
\begin{flalign*}
  f_1 : \tau_1 \rightarrow \tau_1',\ f_2 : \tau_2 \rightarrow \tau_2',\ \ldots,\ f_n : \tau_n \rightarrow \tau_n',\\
  g_1 : \tau_1' \rightarrow \sigma_1,\ g_2 : \tau_2' \rightarrow \sigma_2,\ \ldots,\ g_n : \tau_n' \rightarrow \sigma_n,
\end{flalign*}
the following equality holds:
\begin{gather*}
  \mathcal{F}(g_1 \circ f_1,\ g_2 \circ f_2,\ \ldots,\ g_n \circ f_n) = \mathcal{F}(h_1,\ h_2,\ \ldots,\ h_n) \circ \mathcal{F}(h_1',\ h_2',\ \ldots,\ h_n'),\\
  \text{where } h_i =
  \begin{cases}
    g_i & \text{if } \beta_i = +\\
    f_i & \text{otherwise}
  \end{cases},\ 
  h_i' =
  \begin{cases}
    f_i & \text{if } \beta_i = +\\
    g_i & \text{otherwise}
  \end{cases}.
\end{gather*}
Note that this is an assertion that every type constructor in System $F_\kappa$ must be functorial.\\
}

{
\noindent
\textbf{Proposition 7.} Let $\theta \in List\{+, -\}$ be a list where $n = |\theta|$, and $\upsilon^\theta$ be a higher-kinded type, in System $F_\kappa$. Given a type constructor $T : \upsilon^\theta$, partial application yields another type constructor, given by
\begin{gather*}
T(\tau_1) : *^{\theta_2} \Rightarrow *^{\theta_3} \Rightarrow \ldots \Rightarrow *^{\theta_n} \Rightarrow *,\\ 
T(\tau_1)(\tau_2,\ \tau_3,\ \ldots,\ \tau_n) = T(\tau_1,\ \tau_2,\ \tau_3,\ \ldots,\ \tau_n).
\end{gather*}
The proof that $T(\tau_1)$ is a well-formed type constructor in System $F_\kappa$ is given by the functoriality of $T$.\\
}

{
\noindent
\textbf{Proposition 8.} In System $F_\kappa$, $\rightarrow$ is a type constructor of the form 
\begin{flalign*}
(\rightarrow) : *^- \Rightarrow * \Rightarrow *.
\end{flalign*}
Observe the application of $\rightarrow$ on types:
\begin{flalign*}
(\rightarrow)(\tau_1,\ \tau_2) = \tau_1 \rightarrow \tau_2.
\end{flalign*}
To prove $\rightarrow$ is a type constructor in System $F_\kappa$, it is necessary to show there exists a polymorphic function
\begin{flalign*}
  \mathcal{F} : (\alpha_1 \rightarrow \alpha_2) \rightarrow (\beta_1 \rightarrow \beta_2) \rightarrow (\alpha_2 \rightarrow \beta_1) \rightarrow (\alpha_1 \rightarrow \beta_2),
\end{flalign*}
such that the functorial laws, outlined in Definition 3, are satisfied. Given terms 
\begin{flalign*}
f : \sigma_2 \rightarrow \tau_2,\ g : \sigma_1 \rightarrow \sigma_2,\ h : \tau_1 \rightarrow \sigma_1,
\end{flalign*}  
$\mathcal{F}$ is defined as
\begin{flalign*}
  \mathcal{F}(h, f) = g \mapsto f \circ g \circ h.
\end{flalign*}  
The identity law is shown to be satisfied by
\begin{flalign*}
  \mathcal{F}(\iota_{\sigma_1}, \iota_{\sigma_2}) = g \mapsto \iota_{\sigma_2} \circ g \circ \iota_{\sigma_1} = g \mapsto g.
\end{flalign*}  
Given terms $f' : \tau_2 \rightarrow \rho_2$, $g' : \tau_1 \rightarrow \tau_2$, $h' : \rho_1 \rightarrow \tau_1$,
\begin{flalign*}
  \mathcal{F}(h \circ h', f' \circ f) &= g \mapsto (f' \circ f) \circ g \circ (h \circ h')\\
  &= g \mapsto f' \circ (f \circ g \circ h) \circ h'\\
  &= (g \mapsto f' \circ g' \circ h') \circ (g \mapsto f \circ g \circ h)\\
  &= \mathcal{F}(f', h') \circ \mathcal{F}(f, h), 
\end{flalign*} 
thus, as expected, composition is preserved.\\
}

{
\noindent
\textbf{Definition 9.} Given the Hindley-Milner unification function $U$, define the function $U^*$ as
\begin{flalign*}
  U^*([\ ], [\ ], S) &= [\ ],\\
  U^*([\tau_1,\tau_2,\ldots,\tau_n],[\sigma_1,\sigma_2,\ldots,\sigma_n], S) &= X \mdoubleplus U^*([\tau_2,\ldots,\tau_n],[\sigma_2,\ldots,\sigma_n], Y)\\
  &\quad\ \text{where } X = U(\tau_1, \sigma_1)\\
  &\qquad \qquad \ Y = S \mdoubleplus X.
\end{flalign*}
Let $\theta$, $\gamma \in List\{+, -\}$ be lists, where $n = |\theta|$, $m = |\gamma|$, and $\upsilon^\theta$ and $\omega^\gamma$ be higher-kinded types in System $F_\kappa$. Denote $[\tau_1,\tau_2,\ldots,\tau_n]$ as $[\![\tau]\!]_n$. Given type constructors $T_1 : \upsilon^\theta$, $T_2 : \omega^\gamma$, and polymorphic type constructors $F : \upsilon^\theta$, $G : \omega^\gamma$, $U$ can be extended in System $F_\kappa$ as follows
\begin{flalign*}
  U(T_1\ \tau_1\ \ldots\ \tau_n,\ T_2\ \sigma_1\ \ldots\ \sigma_m) &=
  \begin{cases}
    U^*([\![\tau]\!]_n, [\![\sigma]\!]_n, [\ ]) & \text{if } T_1 = T_2\\
    \bot & \text{otherwise}
  \end{cases},\\
  U(F\ \tau_1\ \ldots\ \tau_n,\ G\ \sigma_1\ \ldots\ \sigma_m) &=
  \begin{cases}
    U^*([\![\tau]\!]_n, [\![\sigma]\!]_n, [F/G]) & \text{if } \theta = \gamma\\
    \bot & \text{otherwise}
  \end{cases},\\
  U(T_1\ \tau_1\ \ldots\ \tau_n,\ F\ \sigma_1\ \ldots\ \sigma_n) &= [T_1/F],\\
  U(F\ \tau_1\ \ldots\ \tau_n,\ T_1\ \sigma_1\ \ldots\ \sigma_n) &= [T_1/F].
\end{flalign*}
}

{
\noindent
\textbf{Remark.} Although the given description of System $F_\kappa$ provides the necessary framework for the desired computation tool, it is far from comprehensive. An exploration of additional properties of System $F_\kappa$ is left to future work.
\par
System $F_\kappa$ describes a generalised functorial calculus, as formulated by McCusker and Santamaria, over the types of the Hindley-Milner type system. Accordingly, further investigation of System $F_\kappa$ would benefit from a description of the categorical semantics which underlies the involved functors and transformations.
}

\end{document}
