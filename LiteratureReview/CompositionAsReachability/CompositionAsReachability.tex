% !TeX root = CompositionAsReachability.tex
\documentclass[../../Dissertation.tex]{subfiles}

\begin{document}
In this section, a functorial calculus, known as the Godement calculus, will be introduced. A generalisation of this calculus, for functors in many-variables, will then be explored. This will lead to a graphical depiction of dinaturality conditions.
\par
In a rather unexpected digression, a model of distributed computation, known as Petri nets, in conjunction with the associated notion of reachability, will be described. Finally, an approach to formulating compositionality of dinatural transformations as reachability in Petri net models, will be explored.

\subsection{The Godement calculus}
\citeasnoun{Godement} outlined five rules for a functorial calculus. Given functors
\begin{flalign}
  F : \mathbb{A} \rightarrow \mathbb{B},\ G : \mathbb{B} \rightarrow \mathbb{C},\\
  H,\ I,\ J : \mathbb{C} \rightarrow \mathbb{D},\\
  K : \mathbb{D} \rightarrow \mathbb{E},\ L : \mathbb{E} \rightarrow \mathbb{F},
\end{flalign}
and natural transformations $\alpha : H \rightarrow I$, $\beta : I \rightarrow J$, as depicted in the diagram
\begin{equation}\label{eq:nat_vertical}
  \begin{tikzpicture}[baseline= (a).base]
    \node[scale=1.2] (a) at (0,0) {
      \begin{tikzcd}
        \mathbb{A} \arrow[r, "F"{name=F, above}]
        &
        \mathbb{B} \arrow[r, "G"{name=G, above}]
        &
        \mathbb{C}
        \arrow[r, bend left=90, "H"{name=LU, above}]
        \arrow[r, "I"{name=LM, below}]
        \arrow[r, bend right=90, "J"{name=LD, below}]
        \arrow[r, Rightarrow, yshift=0.7, shorten <= 2.5pt, from=LU, to=LM, "\hspace{1.8pt}\alpha"{}]
        \arrow[r, Rightarrow, yshift=0.8, shorten <= 1.0pt, from=LM, to=LD, "\hspace{1.8pt}\beta"{}]
        &
        \mathbb{D} \arrow[r, "K"{name=K, above}]
        &
        \mathbb{E} \arrow[r, "L"{name=L, above}]
        &
        \mathbb{F}
      \end{tikzcd},
    };
  \end{tikzpicture}
\end{equation}
Godement asserts the following equational laws must hold:
\begin{flalign}
  \alpha_{GF} &= (\alpha_G)_F,\\
  (LK) * \alpha &= L(K * \alpha),\\
  (K * \alpha)_G &= K(\alpha_G),\\
  K(\beta \circ \alpha)_G &= (K * \beta_G) \circ (K * \alpha_G).
\end{flalign}
The illusive fifth rule is given by the interchange law, as outlined in Section \ref{sec:naturalcomp}. Each of the remaining four rules require understanding the notion of composing a functor with a natural transformation, known as whiskering.
\par
Whiskering is an instance of horizontal composition of 2-cells, in which one 2-cell is taken to be the identity on a 1-cell. By identifying a 1-cell with its associated identity 2-cell, it is possible to formalise the notion of horizontal composition of a 2-cell with a 1-cell. Pre-composing a 2-cell with a 1-cell is termed left whiskering, while post-composing is termed right whiskering.
\par
In the category $Cat$, whiskering translates to a technique for defining the horizontal composition of a natural transformation with a functor. Let $F, G : \mathbb{C} \rightarrow \mathbb{D}$ and $H, K: \mathbb{D} \rightarrow \mathbb{E}$ be functors, and $\alpha : F \rightarrow G, \beta : H \rightarrow K$ be natural transformations. Whiskering $F$ and $\beta$ is achieved by associating $F$ with $id_F : F \rightarrow F$, and is depicted by the following commutative diagram:

\begin{equation}\label{eq:nat_vertical}
  \begin{tikzpicture}[baseline= (a).base]
    \node[scale=1.2] (a) at (0,0) {
      \begin{tikzcd}
        \mathbb{C} \arrow[r, "F"{name=F, above}]
        &
        \mathbb{D}
        \arrow[r, bend left=60, "H"{name=H, above}]
        \arrow[r, swap, bend right=60, "K"{name=K, below}]
        \arrow[r, Rightarrow, from=H, to=K, shorten <= 2.6pt, shorten >= 2.6pt, "\beta"{}]
        &
        \mathbb{E}
      \end{tikzcd}.
    };
  \end{tikzpicture}
\end{equation}
\\
This may also be formulated with the equation $(\beta * F)_X = \beta_{F(X)}$. Whiskering $G$ and $\beta$ can be achieved in an identical fashion; by associating $G$ with the natural transformation $id_G : G \rightarrow G$. Alternatively, whiskering $\alpha$ and $H$ is achieved by associating $H$ with $id_H : H \rightarrow H$, and is depicted by the following commutative diagram:

\begin{equation}\label{eq:nat_vertical}
  \begin{tikzpicture}[baseline= (a).base]
    \node[scale=1.2] (a) at (0,0) {
      \begin{tikzcd}
        \mathbb{C}
        \arrow[r, bend left=60, "F"{name=F, above}]
        \arrow[r, swap, bend right=60, "G"{name=G, below}]
        \arrow[r, Rightarrow, from=F, to=G, shorten <= 2.6pt, shorten >= 2.6pt, "\alpha"{}]
        &
        \mathbb{D} \arrow[r, "H"{name=H, above}]
        &
        \mathbb{E}
      \end{tikzcd}.
    };
  \end{tikzpicture}
\end{equation}
\\
This may also be formulated with the equation $(H * \alpha)_X = H (\alpha_X)$. Once again, whiskering $\alpha$ and $K$ is achieved in an identical manner; associating $K$ with the natural transformation $id_K : K \rightarrow K$. Two equivalent definitions of the horizontal composition of $\alpha$ and $\beta$ are described in terms of whiskering:
\begin{flalign}\label{horizontal-def}
  \beta * \alpha = (\beta * G) \bullet (H * \alpha),\\
  \beta * \alpha = (K * \alpha) \bullet (\beta * F).
\end{flalign}

\subsection{Many-Variable Functorial Calculus}\label{sec:manyvar}
The categorical semantics of the simply-typed lambda calculus, described in \ref{sec:systemfcat}, requires the construction of a category which carries additional structure. The conditions imposed on this structure are concerned with the equality of specific compositions of morphisms, and are termed coherence conditions.
\par
In search of an abstract theory of coherence problems, \citeasnoun{ManyVariableFunctorialCalculus} develops a calculus of substitution; a generalisation of composition in the Godement calculus. In Kelly's many-variable functorial calculus, given functors $F : \mathbb{A}^{op} \times \mathbb{A} \times \mathbb{B} \rightarrow \mathbb{C}$, $G : \mathbb{D} \rightarrow \mathbb{A}$ and $H : \mathbb{E} \rightarrow \mathbb{B}$, the composition $F \cdot \left(G^{op} \times G \times H\right) : \mathbb{D}^{op} \times \mathbb{D} \times \mathbb{E} \rightarrow \mathbb{C}$ may instead be given by the substitution $F(G, G, H)$.
\par
Kelly presents a representation of naturality conditions as graphs, and describes the operations of composition and substitution on these graphs. Given functors $F : \mathbb{C}^{op} \times \mathbb{C} \rightarrow \mathbb{D}$, $G : \mathbb{C} \rightarrow \mathbb{D}$ and a dinatural transformation $\phi : F \rightarrow G$, the naturality condition for $\phi$ is depicted as
\begin{equation}\label{eq:kellyco}
  \begin{tikzpicture}[baseline= (a).base]
    \node[scale=1.2] (a) at (0,0) {
      \begin{tikzcd}
        + \arrow[dash, r] \arrow[dash, d, bend left=75]
        & 
        +
        \\
        - 
        &
      \end{tikzcd}.
    };
  \end{tikzpicture}
\end{equation}
Nodes in (\ref{eq:kellyco}) are labelled with $+$ or $-$, to denote covariant and contravariant arguments to a functor, respectively. Arcs are used to show which arguments are to be set equal.
\par
The composition of dinatural transformations can be depicted  by connecting their associated graphs. Graphs are connected together by pasting the codomain of the source along the domain of the target. This has the potential to create closed cycles within the composite graph, as in
\begin{equation}\label{eq:kellyco}
  \begin{tikzpicture}[baseline= (a).base]
    \node[scale=1.2] (a) at (0,0) {
      \begin{tikzcd}
        + \arrow[dash, r]
        & 
        + \arrow[dash, d, bend right=75]
        \\
        &
        -
      \end{tikzcd}
      $\cdot$
      \begin{tikzcd}
        + \arrow[dash, r] \arrow[dash, d, bend left=75]
        & 
        +
        \\
        - 
        &
      \end{tikzcd}
      $=$  
      \begin{tikzcd}
        + \arrow[dash, r]
        & 
        + \arrow[dash, r] \arrow[dash, d, bend left=75] \arrow[dash, d, bend right=75]
        &
        +
        \\
        &
        -
        &
      \end{tikzcd}.
    };
  \end{tikzpicture}
\end{equation}
The creation of such cycles directly corresponds to the problem of composing dinatural transformations, in general. Substitution corresponds to the operation of replacing arcs with graphs, and does not generate cycles.
\par
Kelly distinguishes the class of graphs in which cycles are prohibited, termed $P^*_0$, from those which permit cycles, termed $P^*$. A formulation of the category $P^*_0$ is given, while a suitable definition of $P^*$ requires a natural method for ordering cycles, which permits substitution, and is left as an open problem.

\subsection{Petri nets}

Petri nets are a mathematical model of distributed computation, first conceptualised by  \citeasnoun{Petri} in his seminal paper Kommunikation mit Automaten. A Petri net consists of weighted, directed arcs between places and transitions. Places may contain tokens and a specific configuration of tokens is known as a marking. Each transition within a Petri net has incoming arcs from input places and outgoing arcs to output places. A transition is enabled when all of its input places contain a sufficient number of tokens, as defined by the weight of the respective incoming arc. Enabled transitions may fire, which results in the tokens in each input place being consumed and each output place being filled with the number of tokens specified by the respective outgoing arc.
\begin{figure}[H]
  \begin{center}
    \begin{tikzpicture}[node distance=1.3cm,>=stealth',bend angle=45,auto]

      \begin{scope}
        \node [place,tokens=2] (i1)                              {};
        \node [place,tokens=1] (i2)  [below of=i1]               {};
        \node [place,tokens=1] (i3)  [below of=i2]               {};

        \node [place]          (o1)  [right of=i1, xshift=15mm]  {};
        \node [place]          (o2)  [right of=i3, xshift=15mm]  {};

        \node [transition]     (t1)  [right of=i2, label=above:$t$] {}
        edge [pre]  node[swap] {2} (i1)
        edge [pre]                 (i2)
        edge [pre]                 (i3)
        edge [post] node[swap] {3} (o1)
        edge [post] node       {2} (o2);
      \end{scope}

      \begin{scope} [xshift=8cm]
        \node [place]          (i1')                              {};
        \node [place]          (i2') [below of=i1']               {};
        \node [place]          (i3') [below of=i2']               {};

        \node [place,tokens=3] (o1') [right of=i1', xshift=15mm]  {};
        \node [place,tokens=2] (o2') [right of=i3', xshift=15mm]  {};

        \node [transition]     (t1') [right of=i2', label=above:$t$] {}
        edge [pre]  node[swap] {2} (i1')
        edge [pre]                 (i2')
        edge [pre]                 (i3')
        edge [post] node[swap] {3} (o1')
        edge [post] node       {2} (o2');
      \end{scope}

      \draw[->,thick] ([xshift=15mm]t1 -| t1) -- ([xshift=-10mm]i2' -| i2')
      node [above=1mm,midway,text width=3cm,text centered]
      {transition $t$ fires};

    \end{tikzpicture}
  \end{center}
  \caption{Firing a transition in a Petri net with three input and two output places.}
  \label{fig:PetriNet}
\end{figure}

Figure \ref{fig:PetriNet} depicts the marking of a Petri net before and after the firing of its single transition $t$. Prior to firing, the input places of $t$ contain the exact number of tokens defined by the incoming arcs (2, 1, 1), therefore enabling $t$. Once $t$ fires, the tokens in its input places are consumed and the outgoing arcs define the number of tokens which are inserted into the output places (3, 2).
\par
Formally, a Petri net $N$ may be expressed as a 4-tuple $(P,T,D,M^0)$ where
\begin{flalign*}
  \hspace{0.75cm}&P \text{ is a finite set of places,}&\\
  &T \text{ is a finite set of transitions,}&\\
  &D \text{ is an incidence matrix of size } |P| \times |T| \text{ (the transition relation),}&\\
  &M^0 \text{ is the initial marking vector; $M^0_p$ is the number of tokens in place $p$.}&
\end{flalign*}
The matrix $D$ of the Petri net $N$ may be defined as follows:
\begin{flalign}
  &&&\ D^{-}_{pt} &&= \text{ the weight of the arc from $p \in P$ to $t \in T$,}&&&\\
  &&&\ D^{+}_{tp} &&= \text{ the weight of the arc from $t \in T$ to $p \in P$,}&&&\\
  &&&\ D^{T} &&= \hspace{0.1cm}D^{+} - D^{-}.&&&\label{eq:transition_relation}
\end{flalign}
Given a marking $M$, a transition $t$ is \textit{enabled} if $\forall p \in P : M_p \geq D_{pt}^-$. Firing $t$ yields the marking $M'$, where $M'_p = M_p + D_{tp}^+ - D_{pt}^-$. This is alternatively denoted $M \xrightarrow[t]{} M'$, which states that $M'$ is reachable from $M$ in one-step. The reflexive transitive closure of $\xrightarrow[t]{}$ is denoted $\xrightarrow[t]{*}$. A marking $M^d$ is said to be reachable from $M$ if $M \xrightarrow[t]{*} M^d$. 
\par
A firing sequence $F = t_1,\ t_2,\ \ldots,\ t_n$ is a sequence of transitions such that $M_0 \xrightarrow[t_1]{} M_2 \xrightarrow[t_2]{} \ldots \xrightarrow[t_n]{} M_n$. Given a firing sequence $F$, there exists a vector $F^{\rightarrow}$, given by
\begin{flalign}
  &&&&&F^{\rightarrow}_{t} &&= \text{ the number of times } t \in T \text{ fires in F,}&\\
  &&&&&D^{T}F^{\rightarrow} + M^{0} &&= \text{ the marking of $N$ after executing sequence $F$.}&
\end{flalign} 

\subsection{Reachability}
Reachability describes a class of computational problems characterised by deciding whether a specific state is reachable given an initial state and a set of permitted transformations. A formulation of this problem for Petri nets is to decide whether a specific marking is reachable in a given net, with a given initial marking. \citeasnoun{ReachibilityConditions} outline the class of Petri nets for which a necessary and sufficient condition for reachability is obtainable. Formally, for a Petri net $N$, described by the 4-tuple $(P,T,D,M^0)$, the set of reachable markings $R_N$ is defined as
\begin{flalign}
  R_N = \{\,M \mid \exists F : M = D^{T}F^{\rightarrow} + M_0 \land F \text{ is a firing sequence of } N\,\}.
\end{flalign}

\citeasnoun{ReachabilityEXPSPACE} shows that the state reachability (coverability) problem for Petri nets is exp-space hard; solvable by a deterministic Turing machine in exponential space. Following this research, an algorithm for deciding state reachability in any given Petri net was derived by \citeasnoun{PetriNetAlgorithm}. Mayr extends the Karp-Miller tree construction for minimal coverability by using non-deterministic finite automata to restrict the transition sequence space.

\citeasnoun{ReachabilityUPPER} derive the best known upper-bound for the reachability problem in vector addition systems (Petri nets); recursive non-primitive cubic-Ackermannian space. \citeasnoun{ReachibilityNotElementary} show that the general reachability problem for Petri nets does not have an elementary lower bound. The lower bound for reachability is shown to be an exponential tower where the height of the tower is an elementary function of the size of the transition space.

\subsection{Composition as Reachability}\label{sec:compasreach}
In their paper, `On Compositionality of Dinatural Transformations', \citeasnoun{DinaturalCompose} make use of Petri net models in order to establish a sufficient condition under which the composition of dinatural transformations preserves dinaturality. 
\par
In the pursuit of a Godement calculus for dinatural transformations, Kelly's many-variable functorial calculus is extended to include transformations where arguments are not linked in pairs. An example of such a transformation, given by McCusker and Santamaria, is the diagonal $\delta : A \rightarrow A \times A$. The coherence properties of such generalised dinatural transformations are formulated as Petri nets. These Petri nets are composed in a similar fashion to the graphs presented by Kelly, i.e., by pasting the codomain of one net along the domain of another. 
\par
Let $F : (\mathbb{C}^{op})^n \times \mathbb{C}^m \rightarrow \mathbb{D}$ and $G : (\mathbb{C}^{op})^{n'} \times \mathbb{C}^{m'} \rightarrow \mathbb{D}$ be functors. Define $[n] = \{i \in \mathbb{N} : i \leq n\}$. The type of a dinatural transformation $\phi : F \rightarrow G$, is given by a natural number $t \in \mathbb{N}$, and two functions $f : [n + m] \rightarrow [t]$, $g : [n' + m'] \rightarrow [t]$. A Petri net model $(P,\ T,\ D,\ M^0)$ of the coherence condition of $\phi$, as described by McCusker and Santamaria, is given by
\begin{flalign}
  |P| = n + m &+ n' + m',\ |T| = t,\\
  \forall i \in [n + m],\ j \in [t]\ :\ D^{-}_{i,j} &= 
  \begin{cases}
    1 & \text{if } i > n \land f(i) = j\\
    0 & \text{otherwise}
  \end{cases},\\
  D^{+}_{j,i} &= 
  \begin{cases}
    1 & \text{if } i \leq n \land f(i) = j\\
    0 & \text{otherwise}
  \end{cases},\\
  \forall i \in [n' + m'],\ j \in [t]\ :\ D^{-}_{i + n + m,j} &=
  \begin{cases}
    1 & \text{if } i \leq n' \land g(i) = j\\
    0 & \text{otherwise}
  \end{cases},\\
  D^{+}_{j,i + n + m} &= 
  \begin{cases}
    1 & \text{if } i > n' \land g(i) = j\\
    0 & \text{otherwise}
  \end{cases},\\
  \forall i \in [n + m + n' + m']\ :\ M^0_i &= 
  \begin{cases}
    1 & \text{if } n < i \leq n + m + n' \\
    0 & \text{otherwise}
  \end{cases},\label{eq:initialmark}\\
  D &= (D^+ - D^-)^T.
\end{flalign}

A Petri net, corresponding to the coherence condition of a dinatural transformation, can be represented graphically. Given functors $F$, $G : \mathbb{C}^{op} \times \mathbb{C} \rightarrow \mathbb{D}$, and a morphism $f \in hom(\mathbb{D})$, the coherence condition of a dinatural transformation $\phi : F \rightarrow G$ is modelled by the Petri net depicted in Figure \ref{fig:dinatpetri}.
\begin{figure}[H]
  \begin{center}
  \begin{tikzcd}[column sep=small]
    {\tikz[baseline=(char.base)]{\node[shape=circle,draw,minimum size=20pt,inner sep=0pt,fill=lightgray] (char) {};}} &  & {\tikz[baseline=(char.base)]{\node[shape=circle,draw,minimum size=20pt,inner sep=0pt,fill=white] (char) {$f$};}} \arrow[ld] \\
    & {\tikz[baseline=(char.base)]{\node[shape=rectangle,draw,minimum size=12pt,inner sep=0pt,fill=black] (char) {};}} \arrow[lu] \arrow[rd] &  \\
    {\tikz[baseline=(char.base)]{\node[shape=circle,draw,minimum size=20pt,inner sep=0pt,fill=lightgray] (char) {$f$};}} \arrow[ru] &  & {\tikz[baseline=(char.base)]{\node[shape=circle,draw,minimum size=20pt,inner sep=0pt,fill=white] (char) {};}}
  \end{tikzcd}
  \xrsquigarrow{transition fires}
  \begin{tikzcd}[column sep=small]
  {\tikz[baseline=(char.base)]{\node[shape=circle,draw,minimum size=20pt,inner sep=0pt,fill=lightgray] (char) {$f$};}} &  & {\tikz[baseline=(char.base)]{\node[shape=circle,draw,minimum size=20pt,inner sep=0pt,fill=white] (char) {};}} \arrow[ld] \\
    & {\tikz[baseline=(char.base)]{\node[shape=rectangle,draw,minimum size=12pt,inner sep=0pt,fill=black] (char) {};}} \arrow[lu] \arrow[rd] &  \\
  {\tikz[baseline=(char.base)]{\node[shape=circle,draw,minimum size=20pt,inner sep=0pt,fill=lightgray] (char) {};}} \arrow[ru] &  & {\tikz[baseline=(char.base)]{\node[shape=circle,draw,minimum size=20pt,inner sep=0pt,fill=white] (char) {$f$};}}
  \end{tikzcd}.
  \end{center}
  \caption{The dinaturality condition of $\phi$.}
  \label{fig:dinatpetri}
\end{figure}
Note that grey circles in Figure \ref{fig:dinatpetri} are used to indicate contravariant arguments to the functors involved. The content of each circle describes the argument to a functor. Each circle either contains $f$ or is empty, corresponding to the identity morphism. A particularly astute reader might observe the similarity between this Petri net model and the dinaturality hexagon (\ref{dinaturality}). Indeed, the initial marking corresponds to the bottom `leg' of this hexagon, while the marking after firing of the transition is given by the top `leg'.
\par
The marking $M^d$, corresponding to the top `leg' of the dinaturality hexagon, is given by
\begin{flalign}
  M^d_i =
  \begin{cases}
    1 & \text{if } \forall j\ :\ D^+_{j,i} = 0\\
    0 & \text{otherwise}
  \end{cases}.
\end{flalign}
McCusker and Santamaria prove that the reachability of $M^d$, from the initial marking $M^0$, is a necessary and condition for satisfying the dinatural coherence property. Intuitively, the reachability of the marking $M^d$ from $M^0$ is equivalent to the condition that the dinaturality hexagon commutes. Furthermore, given the class of Petri nets where each place node has at most one incoming and outgoing arc, \citeasnoun{ReachibilityConditions} prove that acyclicity is a necessary and sufficient condition for the reachability of $M^d$ from $M^0$.
\par
An example of a Petri net containing a cycle, can be obtained by composing two nets corresponding to the dinatural coherence condition. Recall that two Petri nets are composed by pasting the codomain of one along the domain of the other. Let $F, G, H : \mathbb{C}^{op} \times \mathbb{C} \rightarrow \mathbb{D}$ be functors. Given dinatural transformations $\phi : F \rightarrow G$ and $\psi : G \rightarrow H$, whose coherence conditions are depicted in Figure \ref{fig:dinatpetri}, the Petri net corresponding to their composition $\psi \cdot \phi$ is given in Figure \ref{fig:petricycle}.

\begin{figure}[H]
  \begin{center}
  \begin{tikzcd}[column sep=small]
    {\tikz[baseline=(char.base)]{\node[shape=circle,draw,minimum size=20pt,inner sep=0pt,fill=lightgray] (char) {};}} &  & {\tikz[baseline=(char.base)]{\node[shape=circle,draw,minimum size=20pt,inner sep=0pt,fill=white] (char) {$f$};}} \arrow[ld] \\
    & {\tikz[baseline=(char.base)]{\node[shape=rectangle,draw,minimum size=12pt,inner sep=0pt,fill=black] (char) {};}} \arrow[lu] \arrow[rd, color=red] &  \\
    {\tikz[baseline=(char.base)]{\node[shape=circle,draw,minimum size=20pt,inner sep=0pt,fill=lightgray] (char) {};}} \arrow[ru, color=red] &  & {\tikz[baseline=(char.base)]{\node[shape=circle,draw,minimum size=20pt,inner sep=0pt,fill=white] (char) {};}} \arrow[ld, color=red] \\
    & {\tikz[baseline=(char.base)]{\node[shape=rectangle,draw,minimum size=12pt,inner sep=0pt,fill=black] (char) {};}} \arrow[lu, color=red] \arrow[rd] &  \\
    {\tikz[baseline=(char.base)]{\node[shape=circle,draw,minimum size=20pt,inner sep=0pt,fill=lightgray] (char) {$f$};}} \arrow[ru] &  & {\tikz[baseline=(char.base)]{\node[shape=circle,draw,minimum size=20pt,inner sep=0pt,fill=white] (char) {};}}
    \end{tikzcd}
  \end{center}
  \caption{A cycle, highlighted in red, is generated by the composition $\psi \cdot \phi$.}
  \label{fig:petricycle}
\end{figure}
By observing the Petri net given in Figure \ref{fig:petricycle}, it can readibly be seen that the highlighted cycle prevents reachability of the marking $M^d$. Thus, the dinaturality of the composition $\psi \cdot \phi$ does not follow from the dinaturality of $\phi$ and $\psi$. 

\end{document}
