% !TeX root = ANaturalModel.tex
\documentclass[../../Dissertation.tex]{subfiles}

\begin{document}
Dinatural transformations have been investigated as a formal model for parametric polymorphism.

\subsection{Categorical semantics of System F}
A model of the simply-typed lambda calculus in a cartesian closed category was first described by \citeasnoun{CCCLambda}. Furthermore, Lambek shows the simply-typed lambda calculus is the internal language of any cartesian closed category. The notion of an internal language provides a manner in which to translate propositions within a given logic into categorical propositions. The construction described by Lambek extends to the second-order polymorphic lambda calculus (System F).
\par
Types in System F are taken to be objects in a category. The set of morphisms between two types $T$ and $U$ is given by the set of functions of the form $T \Rightarrow U$. The identity morphism is given by the polymorphic identity function. Composition is given by the polymorphic composition function.
\par
To complete the construction of a cartesian closed category, it is necessary to show that the category has finite products and that the functor, $- \times X$, has a right adjoint. The terminal object is given by including a unit type, denoted $()$, in the definition of System F. 
\par
Binary products are also included within the definition of System F. For any types $T$ and $U$, there exists a type $T \times U$ and projections $\pi_{T} : T \times U \Rightarrow T$, $\pi_{U} : T \times U \Rightarrow U$, such that for any type $V$ and functions $f : V \Rightarrow T$, $g : V \Rightarrow U$, there exists a function $h : V \Rightarrow T \times U$, such that the following diagram commutes:

\begin{equation}\label{eq:nat_vertical}
  \begin{tikzpicture}[baseline= (a).base]
    \node[scale=1.2] (a) at (0,0) {
      \begin{tikzcd}
        & 
        V \arrow[dl, swap, "f"] \arrow[dr, "g"] \arrow[d, "h"]
        &
        \\
        T
        & 
        T \times U \arrow[l, "\pi_{T}"] \arrow[r, swap, "\pi_{U}"]
        &
        U
      \end{tikzcd}.
    };
  \end{tikzpicture}
\end{equation}
\\
The right adjoint of $- \times X$ is given by the functor $X \Rightarrow -$, which sends a type $Y$ to the function type $X \Rightarrow Y$, and a natural transformation $\epsilon_X : (X \Rightarrow Y) \times X \rightarrow Y$ given by evaluation. Given any function $f : Z \times X \Rightarrow Y$, there exists the unique function $g : Z \Rightarrow (X \Rightarrow Y) = z \mapsto (x \mapsto f (z, x))$, which makes the following diagram commute:
\begin{equation}\label{eq:nat_vertical}
  \begin{tikzpicture}[baseline= (a).base]
    \node[scale=1.2] (a) at (0,0) {
      \begin{tikzcd}
        (X \Rightarrow Y) \times X \arrow[r, "\epsilon_X"]
        &
        Y
        \\
        Z \times X \arrow[u, "g \times id"] \arrow[ur, swap, "f"]
        &
      \end{tikzcd}.
    };
  \end{tikzpicture}
\end{equation}

\subsection{Functorial Polymorphism}
\citeasnoun{FunctorialPolymorphism} introduce a formal model of parametric polymorphism in which the parametricity property is described by naturality conditions. Given a cartesian closed category $\mathbb{C}$ of primitive types, multi-variance functors of the form $F : (\mathbb{C}^{op})^n \times \mathbb{C}^n \rightarrow \mathbb{C}$ can be interpreted as polymorphic types. Dinatural transformations between

Summary of dinaturality in category of partial equivalence relations, as described by \citeasnoun{DinaturalityInPERS} and how this corresponds to parametric polymorphism.
\newline\newline
Describe the problem of dinatural composition in relation to a model for parametric polymorphism (why is this a concern? would imposing further constraints on the discussed model of parameteric polymorphism, in order to only permit composable dinatural transformations, provide an improvement on the model? what implications are there for functional programming languages?).

\subsection{Substitution and coherence problems}
Discuss the notion of the multi-variable functorial calculus described by \citeasnoun{ManyVariableFunctorialCalculus} -- discussions of functors as substitutions here.

\subsection{Extensions to the functorial polymorphism}
Discuss existing extensions to the functorial polymorphism and the `solutions' to the composition problem in the literature.
\newline\newline
\citeasnoun{StructuralPolymorphism} describe the notion of `structors'; a generalisation of functors which carry morphisms to spans of morphisms.
\newline\newline
\citeasnoun{CanonicalDinatural} define `canonical dinatural transformations', a constrained form of dinatural transformations, and relates them to context-free grammars and the polymorphic operator.
\newline\newline
Lead in to the next section on a new constraint in the literature involving petri nets and reachability.

\subsection{Dinatural composition to reachability}
In their paper, `On Compositionality of Dinatural Transformations', \citeasnoun{DinaturalCompose} make use of Petri net models in order to establish conditions under which the composition of dinatural transformations preserves dinaturality. The concept of modelling dinatural transformations with Petri nets and imposing constraints to ensure compositionality, could be developed into a computational system further exploring this idea.
\newline\newline
Formally state the intended project as a product of the reviewed literature.

\end{document}
