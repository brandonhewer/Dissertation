% !TeX root = ANaturalModel.tex
\documentclass[../../Dissertation.tex]{subfiles}

\tikzset{
  adjust height/.style={minimum height=#1*\pgfkeysvalueof{/pgf/minimum width}},
  adjust width/.style={minimum width=#1*\pgfkeysvalueof{/pgf/minimum height}}
}

\begin{document}
Dinatural transformations have been investigated as a formal model for parametric polymorphism. This investigation has motivated the discovery of various practical results on types, such as parametricity.
\par
The Curry-Howard-Lambek correspondence provides an isomorphism between a typed lambda calculus and a cartesian closed category, and thus enables a categorical semantics for System F.

\subsection{Categorical Semantics of System F}\label{sec:systemfcat}
A model of the simply-typed lambda calculus in a cartesian closed category was first described by \citeasnoun{CCCLambda}. Furthermore, Lambek shows the simply-typed lambda calculus is the internal language of any cartesian closed category. The notion of an internal language provides a manner in which to translate propositions within a given logic into categorical propositions. The construction described by Lambek extends to the second-order polymorphic lambda calculus (System F).
\par
Types in System F are taken to be objects in a category. The set of morphisms between two types $T$ and $U$ is given by the set of functions of the form $T \Rightarrow U$. The identity morphism is given by the polymorphic identity function. Composition is given by the polymorphic composition function.
\par
To complete the construction of a cartesian closed category, it is necessary to show that the category has finite products and that the functor, $- \times X$, has a right adjoint. The terminal object is given by including a unit type, denoted $()$, in the definition of System F. 
\par
Binary products are also included within the definition of System F. For any types $T$ and $U$, there exists a type $T \times U$ and projections $\pi_{T} : T \times U \Rightarrow T$, $\pi_{U} : T \times U \Rightarrow U$, such that for any type $V$ and functions $f : V \Rightarrow T$, $g : V \Rightarrow U$, there exists a function $h : V \Rightarrow T \times U$, such that the following diagram commutes:

\begin{equation}\label{eq:nat_vertical}
  \begin{tikzpicture}[baseline= (a).base]
    \node[scale=1.2] (a) at (0,0) {
      \begin{tikzcd}
        & 
        V \arrow[dl, swap, "f"] \arrow[dr, "g"] \arrow[d, "h"]
        &
        \\
        T
        & 
        T \times U \arrow[l, "\pi_{T}"] \arrow[r, swap, "\pi_{U}"]
        &
        U
      \end{tikzcd}.
    };
  \end{tikzpicture}
\end{equation}
\\
The right adjoint of $- \times X$ is given by the functor $X \Rightarrow -$, which sends a type $Y$ to the function type $X \Rightarrow Y$, and a natural transformation $\epsilon_X : (X \Rightarrow Y) \times X \rightarrow Y$ given by evaluation. Given any function $f : Z \times X \Rightarrow Y$, there exists the unique function $g : Z \Rightarrow (X \Rightarrow Y) = z \mapsto (x \mapsto f (z, x))$, which makes the following diagram commute:
\begin{equation}\label{eq:nat_vertical}
  \begin{tikzpicture}[baseline= (a).base]
    \node[scale=1.2] (a) at (0,0) {
      \begin{tikzcd}
        (X \Rightarrow Y) \times X \arrow[r, "\epsilon_X"]
        &
        Y
        \\
        Z \times X \arrow[u, "g \times id"] \arrow[ur, swap, "f"]
        &
      \end{tikzcd}.
    };
  \end{tikzpicture}
\end{equation}

\subsection{Many-Variable Functorial Calculus}
The categorical semantics of System F, described in \ref{sec:systemfcat}, requires the construction of a category which carries additional structure. The conditions imposed on this structure are concerned with the equality of specific compositions of morphisms, and are termed coherence conditions.
\par
In search of an abstract theory of coherence problems, \citeasnoun{ManyVariableFunctorialCalculus} develops a calculus of substitution; a generalisation of composition in the Godement calculus. In Kelly's many-variable functorial calculus, given functors $F : \mathbb{A}^{op} \times \mathbb{A} \times \mathbb{B} \rightarrow \mathbb{C}$, $G : \mathbb{D} \rightarrow \mathbb{A}$ and $H : \mathbb{E} \rightarrow \mathbb{B}$, the composition $F \cdot \left(G^{op} \times G \times H\right) : \mathbb{D}^{op} \times \mathbb{D} \times \mathbb{E} \rightarrow \mathbb{C}$ may instead be given by the substitution $F(G, G, H)$.
\par
Kelly presents a representation of naturality conditions as graphs, and describes the operations of composition and substitution on these graphs. Given functors $F : \mathbb{C}^{op} \times \mathbb{C} \rightarrow \mathbb{D}$, $G : \mathbb{C} \rightarrow \mathbb{D}$ and a dinatural transformation $\phi : F \rightarrow G$, the naturality condition for $\phi$ is depicted as
\begin{equation}\label{eq:kellyco}
  \begin{tikzpicture}[baseline= (a).base]
    \node[scale=1.2] (a) at (0,0) {
      \begin{tikzcd}
        + \arrow[dash, r] \arrow[dash, d, bend left=75]
        & 
        +
        \\
        - 
        &
      \end{tikzcd}.
    };
  \end{tikzpicture}
\end{equation}
Nodes in (\ref{eq:kellyco}) are labelled with $+$ or $-$, to denote covariant and contravariant arguments to a functor, respectively. Arcs are used to show which arguments are to be set equal.
\par
The composition of dinatural transformations can be depicted  by connecting their associated graphs. Graphs are connected together by pasting the codomain of the source along the domain of the target. This has the potential to create closed cycles within the composite graph, as in
\begin{equation}\label{eq:kellyco}
  \begin{tikzpicture}[baseline= (a).base]
    \node[scale=1.2] (a) at (0,0) {
      \begin{tikzcd}
        + \arrow[dash, r]
        & 
        + \arrow[dash, d, bend right=75]
        \\
        &
        -
      \end{tikzcd}
      $\cdot$
      \begin{tikzcd}
        + \arrow[dash, r] \arrow[dash, d, bend left=75]
        & 
        +
        \\
        - 
        &
      \end{tikzcd}
      $=$  
      \begin{tikzcd}
        + \arrow[dash, r]
        & 
        + \arrow[dash, r] \arrow[dash, d, bend left=75] \arrow[dash, d, bend right=75]
        &
        +
        \\
        &
        -
        &
      \end{tikzcd}.
    };
  \end{tikzpicture}
\end{equation}
The creation of such cycles directly corresponds to the problem of composing dinatural transformations, in general. Substitution corresponds to the operation of replacing arcs with graphs, and does not generate cycles.
\par
Kelly distinguishes the class of graphs in which cycles are prohibited, termed $P^*_0$, from those which permit cycles, termed $P^*$. A formulation of the category $P^*_0$ is given, while a suitable definition of $P^*$ requires a natural method for ordering cycles, which permits substitution, and is left as an open problem.

\subsection{Functorial Polymorphism}
\citeasnoun{FunctorialPolymorphism} introduce a formal model of parametric polymorphism in which the parametricity property is described by naturality conditions. Given an enriched cartesian closed category $\mathbb{C}$ of primitive types, multi-variance functors of the form $(\mathbb{C}^{op})^n \times \mathbb{C}^n \rightarrow \mathbb{C}$ are used to model polymorphic types with $n$ type variables. In this interpretation, terms are given by natural transformations and primitive types are modelled by constant functors.
\par
A deeper intuition of how types may be understood as functors of the form $(\mathbb{C}^{op})^n \times \mathbb{C}^n \rightarrow \mathbb{C}$, is found by examination of prominent examples. The type of the polymorphic identity function, $\forall T.\ T \Rightarrow T$, is modelled as a functor of the form $\mathbb{C}^{op} \times \mathbb{C} \rightarrow \mathbb{C}$. The type of the Y-combinator, $\forall T.\ (T \Rightarrow T) \Rightarrow T$, is modelled as a functor of the form $\mathbb{C}^{op} \times \mathbb{C}^{op} \times \mathbb{C} \times \mathbb{C} \rightarrow \mathbb{C}$, which is dummy in the first argument.
\par
Bainbridge et al. construct a cartesian closed category, $\mathbf{PER}(X)$, of partial-equivalence relations on $X$, in which a morphism is named by each partial recursive function $f$ which preserves relationships i.e., $x\ R\ y \rightarrow f(x)\ R'\ f(y)$. Composition is given by the set-theoretic notion of function composition. A class of composable dinatural transformations, denoted `realizable', on the category $\mathbf{PER}(N)$, is identified. It is then shown that any functor definable in System F is `realizable', and every natural transformation between such functors is indeed also `realizable'. Dinatural transformations within System F, defined using the described syntatic approach, are thus shown to be composable.

\subsection{Extensions to Functorial Polymorphism}
Discuss existing extensions to the functorial polymorphism and the `solutions' to the composition problem in the literature.
\newline\newline
\citeasnoun{StructuralPolymorphism} describe the notion of `structors'; a generalisation of functors which carry morphisms to spans of morphisms.
\newline\newline
\citeasnoun{CanonicalDinatural} define `canonical dinatural transformations', a constrained form of dinatural transformations, and relates them to context-free grammars and the polymorphic operator.
\newline\newline
Lead in to the next section on a new constraint in the literature involving petri nets and reachability.

\subsection{Composition as Reachability}
In their paper, `On Compositionality of Dinatural Transformations', \citeasnoun{DinaturalCompose} make use of Petri net models in order to establish conditions under which the composition of dinatural transformations preserves dinaturality. 

\begin{equation}\label{eq:petri}
  \begin{tikzpicture}[baseline= (a).base]
    \node[scale=1.2] (a) at (0,0) {
      \begin{tikzcd}[column sep=tiny]
        {\tikz[baseline=(char.base)]{\node[shape=rectangle,draw,minimum size=15pt, fill=lightgray] (char) {};}} &  & {\tikz[baseline=(char.base)]{\node[shape=rectangle,draw,minimum size=15pt,adjust height=1,fill=white] (char) {$f$};}} \arrow[ld] \\
        & {\tikz[baseline=(char.base)]{\node[shape=rectangle,draw,minimum size=15pt,aspect=1,fill=black] (char) {};}} \arrow[lu] \arrow[rd] &  \\
        {\tikz[baseline=(char.base)]{\node[shape=rectangle,draw,minimum size=15pt,aspect=1,fill=lightgray] (char) {$f$};}} \arrow[ru] &  & {\tikz[baseline=(char.base)]{\node[shape=rectangle,draw,minimum size=15pt,fill=white] (char) {};}}
      \end{tikzcd}
      };
\end{tikzpicture}
\end{equation}

\end{document}
