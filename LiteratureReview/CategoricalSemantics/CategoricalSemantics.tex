% !TeX root = CategoricalSemantics.tex
\documentclass[../../Dissertation.tex]{subfiles}

\begin{document}
In the pursuit of a suitable categorical semantics for parametricity, a well-known isomorphism between formal proof systems, category theory and  programs, will be introduced. This will facilitate the exploration of a composable family of dinatural transformations, which are identified by means of cut-free proofs.
\par
Subsequently, a categorical semantics of the simply-typed lambda calculus will be introduced, providing a category-theoretic model of simple types. Finally, a categorical semantics for polymorphic types in System F will be explored, by means of modelling parametrically polymorphic functions as a composable class of dinatural transformations.

\subsection{Curry-Howard-Lambek correspondence}\label{sec:curryhowardlambek}
A sequent calculus is an inference system for first-order logic. The most basic construct in a sequent calculus, known as a sequent, has the form
\begin{flalign}\label{eq:sequent}
  X_1,\ X_2,\ \ldots,\ X_N \vdash Y_1,\ Y_2,\ \ldots,\ Y_M.
\end{flalign}
A sequent of the form expressed in (\ref{eq:sequent}) is a judgement which asserts that a statement $Y_i$ is true, if $X_1 \land X_2 \land ... \land X_N$ is true. 
\par
Constructs in formal proof systems, such as a sequent calculus, are in direct correspondence with types within a typed lambda calculus, by the Curry-Howard isomorphism \cite{CurryHoward}. A sequent of the form
\begin{flalign}\label{eq:sequent}
  X_1,\ X_2,\ \ldots,\ X_N \vdash Y
\end{flalign}
corresponds to a term of the type $X_1 \times X_2 \times ... \times X_N \Rightarrow Y$, and constructing such a term is equivalent to the construction of a proof.
\par
Connections between intuitionistic logic and category theory, discovered by Lambek \citeyear{LambekCorrespondence,LambekCorrespondence2,LambekCorrespondence3} and \citeasnoun{LawvereCorrespondence}, further extends this relationship to a three-way isomorphism known as the Curry-Howard-Lambek correspondence. The Lambek-Lawvere correspondence can be described as modelling formulae in an intuitionistic propositional logic as objects in a cartesian-closed category, and proofs as morphisms.
\par
Let $\Gamma$, $\Sigma$, $\Delta$ and $\Lambda$ be finite lists of formulae, known as a context. The cut-rule is then defined as follows:

\begin{equation}\label{eq:cut}
\begin{prooftree}
  \hypo{\Gamma \vdash X, \Delta}
  \hypo{\Sigma, X \vdash \Lambda}
  \infer2[cut]{\Sigma, \Gamma \vdash \Lambda, \Delta}
\end{prooftree}.
\end{equation}
The cut-rule depicted in (\ref{eq:cut}) can be described as cutting any occurrence of $X$ from the proof. By the Curry-Howard isomorphism, the cut rule corresponds to a familiar substitution process; function composition. Given a context $\Pi$ in a typed lambda calculus, this substitution may be stated formally as
\begin{equation}\label{eq:lambdacompose}
\begin{prooftree}
  \hypo{\Pi, x : A \vdash t : B}
  \hypo{\Pi, y : B \vdash t' : C}
  \infer2[]{\Pi, x : A \vdash t'[t/y] : C}
\end{prooftree}.
\end{equation}
\par
The cut-elimination theorem, originally proved by \citeasnoun{CutFree}, states that any sequent which has a proof using the cut-rule, also has a proof which does not. Cut-free proofs are typically more complex and larger than alternative proofs which make use of the cut rule. Cut-elimination corresponds to the strong-normalization property of a typed lambda calculus.
\par
The substitution depicted in (\ref{eq:lambdacompose}) reveals a property of the cut-rule; covariant occurrences of proofs are substituted into contravariant occurrences. Conversely, the identity sequent maps contravariant occurrences of proofs to covariant occurrences. 
\par
\citeasnoun{NaturalAsNormalForms} describe how the symmetry induced by the co-existence of the identity sequent and the cut rule is broken by the cut-elimination theorem. This asymmetry is used to show that cut-elimination, when interpreted by dinatural transformations, resolves the cyclic dependencies created by `linking' covariant and contravariant terms. Using cut-free sequent rules, Girard, Scedrov and Scott show that syntactic dinatural transformations between definable functors on a cartesian-closed category, whose internal language is a typed lambda calculus, are composable.

\subsection{Categorical Semantics of System F}\label{sec:systemfcat}
A model of the simply-typed lambda calculus in a cartesian closed category was first described by \citeasnoun{CCCLambda}. Furthermore, Lambek shows the simply-typed lambda calculus is the internal language of any cartesian closed category. The notion of an internal language provides a manner in which to translate propositions within a given logic into categorical propositions. The construction described by Lambek extends to the second-order polymorphic lambda calculus (System F).
\par
Types in the simply-typed calculus are taken to be objects in a category. The set of morphisms between two types $T$ and $U$ is given by the set of functions of the form $T \Rightarrow U$. The identity morphism on $T$ is given by the identity function $\lambda x^T.x$. Composition is given by the polymorphic composition function.
\par
To complete the construction of a cartesian closed category, it is necessary to show that the category has finite products and that the functor, $- \times X$, has a right adjoint. The terminal object is given by including a unit type, denoted $()$, in the definition of the simply-typed lambda  calculus. 
\par
Binary products are also included within the definition. For any types $T$ and $U$, there exists a type $T \times U$ and projections $\pi_{T} : T \times U \Rightarrow T$, $\pi_{U} : T \times U \Rightarrow U$, such that for any type $V$ and functions $f : V \Rightarrow T$, $g : V \Rightarrow U$, there exists a function $h : V \Rightarrow T \times U$, such that the following diagram commutes:

\begin{equation}\label{eq:nat_vertical}
  \begin{tikzpicture}[baseline= (a).base]
    \node[scale=1.2] (a) at (0,0) {
      \begin{tikzcd}
        & 
        V \arrow[dl, swap, "f"] \arrow[dr, "g"] \arrow[d, "h"]
        &
        \\
        T
        & 
        T \times U \arrow[l, "\pi_{T}"] \arrow[r, swap, "\pi_{U}"]
        &
        U
      \end{tikzcd}.
    };
  \end{tikzpicture}
\end{equation}
\\
The right adjoint of $- \times X$ is given by the functor $X \Rightarrow -$, which sends a type $Y$ to the function type $X \Rightarrow Y$, and a natural transformation $\epsilon_X : (X \Rightarrow Y) \times X \rightarrow Y$ given by evaluation. Given any function $f : Z \times X \Rightarrow Y$, there exists the unique function $g : Z \Rightarrow (X \Rightarrow Y) = z \mapsto (x \mapsto f (z, x))$, which makes the following diagram commute:
\begin{equation}\label{eq:nat_vertical}
  \begin{tikzpicture}[baseline= (a).base]
    \node[scale=1.2] (a) at (0,0) {
      \begin{tikzcd}
        (X \Rightarrow Y) \times X \arrow[r, "\epsilon_X"]
        &
        Y
        \\
        Z \times X \arrow[u, "g \times id"] \arrow[ur, swap, "f"]
        &
      \end{tikzcd}.
    };
  \end{tikzpicture}
\end{equation}

\subsection{Functorial Polymorphism}\label{sec:functorial}
\citeasnoun{FunctorialPolymorphism} introduce a formal model of parametric polymorphism in which the parametricity property is described by naturality conditions. Given an enriched cartesian closed category $\mathbb{C}$ of primitive types, multi-variance functors of the form $(\mathbb{C}^{op})^n \times \mathbb{C}^n \rightarrow \mathbb{C}$ are used to model polymorphic types with $n$ type variables. In this interpretation, terms are given by natural transformations and primitive types are modelled by constant functors.
\par
A deeper intuition of how types may be understood as functors of the form $(\mathbb{C}^{op})^n \times \mathbb{C}^n \rightarrow \mathbb{C}$, is found by examination of prominent examples. The type of the polymorphic identity function, $\forall T.\ T \Rightarrow T$, is modelled as a functor of the form $F : \mathbb{C}^{op} \times \mathbb{C} \rightarrow \mathbb{C}$. Given objects $X,\ Y \in \mathbb{C}$ and functions $f : X' \rightarrow X$, $g : Y \rightarrow Y'$, $F$ is defined as follows: 
\begin{flalign}
  F (X, Y) = X \rightarrow Y,\\
  F (f, g) = \lambda h^{X \rightarrow Y}. g \circ h \circ f.
\end{flalign}
The type of the Y-combinator, $\forall T.\ (T \Rightarrow T) \Rightarrow T$, is modelled as a functor of the form $\mathbb{C}^{op} \times \mathbb{C}^{op} \times \mathbb{C} \times \mathbb{C} \rightarrow \mathbb{C}$, which is dummy in the first argument.
\par
Bainbridge et al. construct a cartesian closed category, $\mathbf{PER}(X)$, of partial-equivalence relations on $X$, in which a morphism is named by each partial recursive function $f$ which preserves relationships i.e., $x\ R\ y \rightarrow f(x)\ R'\ f(y)$. Composition is given by the set-theoretic notion of function composition. A class of composable dinatural transformations, denoted `realizable', on the category $\mathbf{PER}(X)$, is identified. It is then shown that any functor definable in System F is `realizable', and every natural transformation between such functors is indeed also `realizable'. Dinatural transformations within System F, defined using the described syntactic approach, are thus shown to be composable.
\par
\citeasnoun{DinaturalInSystemF} formulates a sufficient condition for dinaturality, in both the Church and Curry-style second-order polymorphic lambda calculus. Lataillade first describes the syntactic category $\mathcal{T}$ for System F, in conjunction with
functors of the form $F[(\_,\_),X] : \mathcal{T}^{op} \times \mathcal{T} \rightarrow \mathcal{T}$. Each functor $F[(A,B),X]$ represents a substitution in which contravariant occurrences of the type variable $X$ are replaced with $A$ and covariant occurrences are replaced with $B$. The dinaturality condition between these substitution functors are used to create a sufficient condition for dinaturality in System F. In the Church-style second-order polymorphic lambda calculus, a term $T$ is shown to be dinatural in a type variable $X$ if and only if $X$ is not in the set of free type variables across all type instantiations of the normal-form of $T$.
\par
Two significant results have been explored. The first, introduced by Bainbridge et al., provides a formulation of types, in System F, as mixed-variance functors. Furthermore, a class of composable dinatural transformations between these mixed-variance functors is identified. The second result, introduced by de. Lataillade, describes a sufficient condition for dinaturality in System F. Although these results establish the foundations of a formal relationship between dinaturality and parametricity, the precise details of the nature of this relationship remain unclear.
\par
Identifying when dinatural transformations, in general, do not compose, will highlight when dinaturality conditions are too weak a formalisation to model parametericity. Thus, a mixed-variance functorial calculus will be sought, in the pursuit of a necessary condition for the composition of dinatural transformations.

\end{document}
