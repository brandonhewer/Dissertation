% !TeX root = Design.tex
\documentclass[../Dissertation.tex]{subfiles}

\begin{document}

\section{System Design}
A formal approach to constructing Petri nets from types is developed in Section \ref{sec:towardsatool}. This formalisation provides the theoretical foundation for a computational tool. However, the development of a functional software application, requires determining and designing specific implementation details. Examples of implementation details include formal specifications of user input, design of graphical components, and the choice of programming language. 
\par
The proposed software tool will be a desktop GUI application. This application should facilitate the creation and composition of Petri nets from types in System $F_\kappa$. The design of key features and graphical components, of the desired software application, will  be detailed in this section.

\subsection{Formal grammars}
The proposed application requires a means by which to accept and interpret an input type from a user. Types are to be provided in a textual format. In order to interpret the textual input, a type parser will be required. A parser requires a formal depiction of a context-free grammar for types in System $F_\kappa$. Backus-Naur form (BNF) is a typical, and appropriate notation for describing a context-free grammar. Accordingly, a formal grammar on types, expressed in BNF, is given in Figure \ref{fig:typegrammar}.

\begin{figure}[H]
\subfile{DataTypeGrammar}
\caption{Formal grammar for types in System $F_\kappa$, expressed in BNF.}
\label{fig:typegrammar}
\end{figure}

The grammar on types depicted in Figure \ref{fig:typegrammar}, introduces \lstinline{=>} as a terminal. This terminal is used to highlight the functors involved in the corresponding dinatural transformation. For example, the string \lstinline{a -> a => a} corresponds to a dinatural transformation between the $\rightarrow$ functor and the identity functor. The \lstinline{=>} terminal is of the lowest precedence in the provided grammar.
\par
The variance of an argument provided to a type constructor is specified by appending either a \lstinline{+} or a \lstinline{-}. If neither is appended, positive variance (covariance) is assumed. For example, the type \lstinline{F a a- a- a} is a valid functor in four arguments, which is covariant in the first and last and contravariant in the second and third. The type constructor \lstinline{F}, corresponds to the higher-kinded type $* \Rightarrow *^- \Rightarrow *^- \Rightarrow *$ in System $F_\kappa$. The type \lstinline{F a a- a- a} is unifiable with both \lstinline{(a -> a)} \lstinline{-> a -> a} and \lstinline{F (a -> a)- (a -> a)}.
\par
Notably, there is no means by which to denote universal quantification or any primitive type, within the provided grammar. Indeed, every identifier is implicitly treated as a type variable which has been universally quantified. This also remains true for functor identifiers. As such, every type and type constructor is considered to be parametrically polymorphic. While this may appear to limit which Petri nets can be generated, this is not the case. This supposed problem will be explored further in Section \ref{sec:petrinetdesign}.
\par
Another noteworthy feature of the provided grammar, is the absence of an explicit representation of a product type. This choice simplifies the syntax while not impacting the range of expressible types. Example representations of a product type include \lstinline{Pair a b} or simply \lstinline{F a b}. 
\par
In addition to a System $F_\kappa$ type, a user is also expected to provide input which describes a naturality type. A naturality type is defined by two  functions. The first of these functions maps each type variable, in the domain of a transformation, to a number. The second function operates identically on the codomain. Any representation of a naturality type should associate to every type variable, a number.
\par
Returning to an earlier example, given in Listing 8, consider the function
\begin{center}
\begin{tabular}{c}
\begin{lstlisting}
applyToFirst :: (a -> a, a, a) -> (a, a).
\end{lstlisting}
\end{tabular}
\end{center}
The type of \lstinline{applyToFirst} may be expressed as a transformation, in the outlined grammar, as
\begin{center}
\begin{tabular}{c}
\begin{lstlisting}
F (a -> a) a a => G a a.
\end{lstlisting}
\end{tabular}
\end{center}
An intuitive depiction of a corresponding naturality type can be attained by writing the same term, and replacing each type variable with the number which it is mapped to. The default naturality type, where every occurrence of \lstinline{a} is equated, can therefore be expressed as 
\begin{center}
\begin{tabular}{c}
\begin{lstlisting}
F (1 -> 1) 1 1 => G 1 1.
\end{lstlisting}
\end{tabular}
\end{center}
Within this depiction of a naturality type, the functor identifiers \lstinline{F} and \lstinline{G} provide redundant information. This naturality type can therefore simply be expressed as
\begin{center}
\begin{tabular}{c}
\begin{lstlisting}
(1 -> 1) 1 1 => 1 1.
\end{lstlisting}
\end{tabular}
\end{center}
The more descriptive coherence condition on \lstinline{applyToFirst}, depicted in Figure \ref{fig:applyToFirstBetter}, is described by the naturality type
\begin{center}
\begin{tabular}{c}
\begin{lstlisting}
(1 -> 1) 1 2 => 1 2.
\end{lstlisting}
\end{tabular}
\end{center}
\par
The formal grammar for the described representation of a naturality type is provided in Figure \ref{fig:natgrammar}.

\begin{figure}[H]
\subfile{NaturalityTypeGrammar}
\caption{Formal grammar for naturality types, expressed in BNF.}
\label{fig:natgrammar}
\end{figure}

\subsection{Petri net design}\label{sec:petrinetdesign}
A graphical representation of a Petri net, within the proposed application, must illustrate key information relating to the corresponding coherence condition. This includes the variance of arguments to the involved functors. Multiple depictions of Petri nets have been given throughout previous sections, and these will be drawn upon in the following design.
\par
Consider a place node within a Petri net. For the purposes of the proposed application, a place node, and its contents, correspond to an argument to a functor. A place node may be graphically depicted as a circle. However, the variance of the functor argument cannot be determined by this simple representation. This problem is resolved by depicting a place node corresponding to a contravariant argument as a grey circle. A place node may either contain one or zero tokens. An empty place node corresponds to an identity argument. A token is depicted with the letter $f$, denoting a function.
\par
In order to distinguish transition nodes from place nodes, a different shape will be used. Transition nodes will be graphically depicted as a square. Recall that a transition node may either be enabled or disabled. To reflect the two possible states, it is necessary to provide a depiction of a disabled transition node. A disabled transition node will be depicted as a black square. 
\par
Clicking an enabled transition node, in the proposed application, should fire that transition. This operation will consume the tokens from the input places of the transition, and push tokens to the output places. The action of firing a transition node corresponds to applying the associated coherence condition.
\par
The described graphical depiction of Petri nets is best understood by means of example. Consider the concatenation function, whose type, in Haskell, is given by
\begin{center}
\begin{tabular}{c}
\begin{lstlisting}
(++) :: [a] -> [a] -> [a].
\end{lstlisting}
\end{tabular}
\end{center}
The type of \lstinline{(++)}  may be expressed as a transformation, within the proposed application, as
\begin{center}
\begin{tabular}{c}
\begin{lstlisting}
List a => List a -> List a.
\end{lstlisting}
\end{tabular}
\end{center}
Given the naturality type,
\begin{center}
\begin{tabular}{c}
\begin{lstlisting}
1 => 1 -> 1,
\end{lstlisting}
\end{tabular}
\end{center}
the corresponding Petri net for \lstinline{(++)}, formally given by 
\begin{equation*}
\mathcal{P}([List\ a,\ List\ a \rightarrow List\ a],\ [n \mapsto 1],\ [n \mapsto 1]),
\end{equation*}
is graphically depicted in Figure \ref{fig:concatpetri}.
\begin{figure}[H]
  \begin{center}
    \begin{tikzcd}
    & {\tikz[baseline=(char.base)]{\node[shape=circle,draw,minimum size=20pt,inner sep=0pt,fill=white] (char) {$f$};}} \arrow[d] &  \\
    & {\tikz[baseline=(char.base)]{\node[shape=rectangle,draw,minimum size=12pt,inner sep=0pt,fill=gray] (char) {};}} \arrow[rd] &  \\
    {\tikz[baseline=(char.base)]{\node[shape=circle,draw,minimum size=20pt,inner sep=0pt,fill=lightgray] (char) {$f$};}} \arrow[ru] &  & {\tikz[baseline=(char.base)]{\node[shape=circle,draw,minimum size=20pt,inner sep=0pt,fill=white] (char) {};}}
    \end{tikzcd}
  \end{center}
  \caption{Petri net for \lstinline{(++)}.}
  \label{fig:concatpetri}
\end{figure}
Firing the transition node in Figure \ref{fig:concatpetri} should yield the Petri net given in Figure \ref{fig:concatpetrif}.
\begin{figure}[H]
  \begin{center}
    \begin{tikzcd}
    & {\tikz[baseline=(char.base)]{\node[shape=circle,draw,minimum size=20pt,inner sep=0pt,fill=white] (char) {};}} \arrow[d] &  \\
    & {\tikz[baseline=(char.base)]{\node[shape=rectangle,draw,minimum size=12pt,inner sep=0pt,fill=black] (char) {};}} \arrow[rd] &  \\
    {\tikz[baseline=(char.base)]{\node[shape=circle,draw,minimum size=20pt,inner sep=0pt,fill=lightgray] (char) {};}} \arrow[ru] &  & {\tikz[baseline=(char.base)]{\node[shape=circle,draw,minimum size=20pt,inner sep=0pt,fill=white] (char) {$f$};}}
    \end{tikzcd}
  \end{center}
  \caption{Petri net for \lstinline{(++)} (after firing).}
  \label{fig:concatpetrif}
\end{figure}
Given a function \lstinline{f :: a -> b}, this Petri net corresponds to the following free theorem:
\begin{center}
\begin{tabular}{c}
\begin{lstlisting}
map f xs ++ map f ys = map f (xs ++ ys).
\end{lstlisting}
\end{tabular}
\end{center}
Observe that place nodes are partitioned into separate rows. The top and bottom rows correspond to the domain and codomain, respectively. Partitioning the diagram in this manner provides a visual distinction between the functors involved. This visual distinction will be particularly important in composite Petri nets, where several partitioned domains will be involved.
\par
For the purposes of demonstrating the graphical depiction of a composite Petri net, a second example will be given. Consider the identity function, whose type, in Haskell, is given by 
\begin{center}
\begin{tabular}{c}
\begin{lstlisting}
  id :: a -> a.
\end{lstlisting}
\end{tabular}
\end{center}
In the syntax of the proposed application, the type of \lstinline{id}, is expressible as the transformation \lstinline{a => a}.  The naturality type for \lstinline{id} is \lstinline{1 => 1}. The corresponding Petri net of \lstinline{id}, formally given by
\begin{equation*}
  \mathcal{P}([a,\ a],\ [n \mapsto 1],\ [n \mapsto 1]),
\end{equation*}
is graphically depicted in Figure \ref{fig:petriid}.
\begin{figure}[H]
  \begin{center}
    \begin{tikzcd}
      {\tikz[baseline=(char.base)]{\node[shape=circle,draw,minimum size=20pt,inner sep=0pt,fill=white] (char) {$f$};}} \arrow[d] \\
      {\tikz[baseline=(char.base)]{\node[shape=rectangle,draw,minimum size=12pt,inner sep=0pt,fill=gray] (char) {};}} \arrow[d] \\
      {\tikz[baseline=(char.base)]{\node[shape=circle,draw,minimum size=20pt,inner sep=0pt,fill=white] (char) {};}}
    \end{tikzcd}
  \end{center}
  \caption{Petri net for \lstinline{id}.}
  \label{fig:petriid}
\end{figure}
Composing the transformations \lstinline{id} and \lstinline{(++)}, simply yields \lstinline{(++)}. By application of Definition \ref{def:petricomposition}, the Petri net generated by their composition is graphically depicted in Figure \ref{fig:petriidconcat}.
\begin{figure}[H]
  \begin{center}
    \begin{tikzcd}
      & {\tikz[baseline=(char.base)]{\node[shape=circle,draw,minimum size=20pt,inner sep=0pt,fill=white] (char) {$f$};}} \arrow[d] &  \\
      & {\tikz[baseline=(char.base)]{\node[shape=rectangle,draw,minimum size=12pt,inner sep=0pt,fill=black] (char) {};}} \arrow[rd] &  \\
     {\tikz[baseline=(char.base)]{\node[shape=circle,draw,minimum size=20pt,inner sep=0pt,fill=lightgray] (char) {};}} \arrow[ru] &  & {\tikz[baseline=(char.base)]{\node[shape=circle,draw,minimum size=20pt,inner sep=0pt,fill=white] (char) {};}} \arrow[d] \\
     {\tikz[baseline=(char.base)]{\node[shape=rectangle,draw,minimum size=12pt,inner sep=0pt,fill=gray] (char) {};}} \arrow[u] &  & {\tikz[baseline=(char.base)]{\node[shape=rectangle,draw,minimum size=12pt,inner sep=0pt,fill=black] (char) {};}} \arrow[d] \\
     {\tikz[baseline=(char.base)]{\node[shape=circle,draw,minimum size=20pt,inner sep=0pt,fill=lightgray] (char) {$f$};}} \arrow[u] &  & {\tikz[baseline=(char.base)]{\node[shape=circle,draw,minimum size=20pt,inner sep=0pt,fill=white] (char) {};}}
     \end{tikzcd}
  \end{center}
  \caption{Petri net generated by the composition \lstinline{id . (++)}.}
  \label{fig:petriidconcat}
\end{figure}
Petri nets, which are generated by composition, preserve the composition boundary. This can be observed in the Petri net corresponding to the composition \lstinline{id . (++)}. Furthermore, the diagram preserves the previous partitioning of place nodes, thus illustrating the functors involved.
\par
The graphical representation of Petri nets, in the proposed application, should permit a set of key animations. Scaling will be necessary to view large graphs, which may be generated by multiple compositions. Translation of a Petri net, within its local frame, will be required for focusing on particular components of large graphs. Any animation applied to the graphical representation of a Petri net, must preserve the partitioning of place nodes.

\subsection{Workspace design}\label{sec:workspacedesign}
The proposed application should present an open workspace area to the user. This design choice will facilitate flexible investigation of types and their corresponding Petri nets. Within the workspace, it should be possible to create, edit and compose existing components. Accordingly, it is necessary to provide the design for these interactive components.
\par
Obtaining a dinatural transformation and a naturality type, for the purpose of generating a Petri net, requires a mechanism by which a user can provide the relevant input. A straightforward approach would involve presenting two independent text input fields. Recall that given a transformation, the default naturality type maps all type variables to the same value. For example, given a transformation 
\begin{center}
\begin{tabular}{c}
\begin{lstlisting}
(a -> a) => a -> a,
\end{lstlisting}
\end{tabular}
\end{center}
the default naturality type is
\begin{center}
\begin{tabular}{c}
\begin{lstlisting}
(1 -> 1) => 1 -> 1.
\end{lstlisting}
\end{tabular}
\end{center}
Accordingly, the overhead of creating a naturality type may be eliminated by providing the default naturality type as an initial value. The values in the generated naturality type may then be edited by a user.
\par
Thus far, in the design considerations, only types containing a single type variable have been studied. Consider the Haskell function
\begin{center}
\begin{tabular}{c}
\begin{lstlisting}
  zip :: [a] -> [b] -> [(a, b)].
\end{lstlisting}
\end{tabular}
\end{center}
The type of \lstinline{zip} may be expressed as a transformation, in the proposed application, as
\begin{center}
\begin{tabular}{c}
\begin{lstlisting}
  List a => List b -> List (Pair a b).
\end{lstlisting}
\end{tabular}
\end{center}
This transformation is natural in \lstinline{a} and extranatural in \lstinline{b}. A user should be able to explore the different naturality conditions in each type variable. This requires providing the type variable as an additional input. For example, given the \lstinline{zip} transformation, with naturality type
\begin{center}
\begin{tabular}{c}
\begin{lstlisting}
  1 => 1 -> (1 1),
\end{lstlisting}
\end{tabular}
\end{center}
the Petri net corresponding to the naturality condition on \lstinline{a}, is given by
\begin{equation*}
  \mathcal{P}_a([List\ a,\ List\ b \rightarrow List\ (Pair\ a\ b)],\ [n \mapsto 1],\ [n \mapsto 1]).
\end{equation*}
This Petri net is identical to that of the \lstinline{id} transformation, depicted in Figure \ref{fig:petriid}. In contrast, the Petri net corresponding to the extranaturality condition on \lstinline{b}, given by
\begin{equation*}
  \mathcal{P}_b([List\ a,\ List\ b \rightarrow List\ (Pair\ a\ b)],\ [n \mapsto 1],\ [n \mapsto 1]),
\end{equation*}
is depicted in Figure \ref{fig:petrizipb}.
 
\begin{figure}[H]
  \begin{center}
    \begin{tikzcd}
      & {\tikz[baseline=(char.base)]{\node[shape=rectangle,draw,minimum size=12pt,inner sep=0pt,fill=gray] (char) {};}} \arrow[rd] &  \\
     {\tikz[baseline=(char.base)]{\node[shape=circle,draw,minimum size=20pt,inner sep=0pt,fill=lightgray] (char) {$f$};}} \arrow[ru] &  & {\tikz[baseline=(char.base)]{\node[shape=circle,draw,minimum size=20pt,inner sep=0pt,fill=white] (char) {};}}
     \end{tikzcd}
  \end{center}
  \caption{Petri net for the type variable \lstinline{b} in \lstinline{zip}.}
  \label{fig:petrizipb}
\end{figure}
This example highlights the requirement for three user inputs: a System $F_\kappa$ type, a naturality type, and a type variable. The graphical subcomponents corresponding to these inputs will be grouped into a single interface. An initial design for this interface is given in Figure \ref{fig:inputdesign}.

\begin{figure}[H]
\begin{adjustbox}{width=\columnwidth/3,center}
\centering
\begin{tabu}{|l|l|}
\taburulecolor{gray}
\tabucline[4pt]{1-8}
\taburulecolor{black}
\lstinline{a -> a => a} & \multirow{2}{*}{\lstinline{a}} \\ \cline{1-1}
\lstinline{1 -> 1 => 1} &  \\ \hline
\end{tabu}
\end{adjustbox}
\caption{Design for the user input interface.}
\label{fig:inputdesign}
\end{figure}
The design for the user input interface includes three text fields. The top-left  field is designed to accept a System $F_\kappa$ type, in the grammar outlined in Figure \ref{fig:typegrammar}. The bottom-left field is designed to accept a naturality type, in the grammar outlined in Figure \ref{fig:natgrammar}. Upon change of the System $F_\kappa$ type, the naturality type should be set to the default. The field on the right is designed to accept the type variable. The thick grey border outlining the top edge of the interface depicts a draggable handle. This draggable handle will be used for moving input components within the workspace.
\par
Notably absent from the current design model is the spatial relation between the input interface and the generated Petri net. The Petri net generated by an input interface is to be displayed directly underneath, as depicted in Figure \ref{fig:componentdesign}. The presented draggable handle should move the entire component, within the workspace.
\begin{figure}[H]
\begin{adjustbox}{width=\columnwidth/3,center}
\centering
\begin{tabu}{|l|l|}
\taburulecolor{gray}
\tabucline[4pt]{1-8}
\taburulecolor{black}
\lstinline{a -> a => a -> a} & \multirow{2}{*}{\lstinline{a}} \\ \cline{1-1}
\lstinline{1 -> 1 => 1 -> 1} &  \\ \hline
\end{tabu}
\end{adjustbox}
\begin{center}
\begin{tikzcd}[column sep=small]
{\tikz[baseline=(char.base)]{\node[shape=circle,draw,minimum size=15pt,inner sep=0pt,fill=lightgray] (char) {};}} &  & {\tikz[baseline=(char.base)]{\node[shape=circle,draw,minimum size=15pt,inner sep=0pt,fill=white] (char) {$f$};}} \arrow[ld] \\
  & {\tikz[baseline=(char.base)]{\node[shape=rectangle,draw,minimum size=10pt,inner sep=0pt,fill=gray] (char) {};}} \arrow[lu] \arrow[rd] &  \\
{\tikz[baseline=(char.base)]{\node[shape=circle,draw,minimum size=15pt,inner sep=0pt,fill=lightgray] (char) {$f$};}} \arrow[ru] &  & {\tikz[baseline=(char.base)]{\node[shape=circle,draw,minimum size=15pt,inner sep=0pt,fill=white] (char) {};}}
\end{tikzcd}
\end{center}
\caption{Design for an interactive Petri net component.}
\label{fig:componentdesign}
\end{figure}

\subsection{Language and Tooling}
C\lstinline{++}17 will be used for implementing the parsing and composition of types in addition to the generation of in-memory representations of Petri nets. The C\lstinline{++} programming language provides a range of low and high-level programming constructs. Low-level concepts such as manual memory-management and move semantics provide opportunities for improving the performance of the application. High-level concepts such as lambdas and generics, in the form of templates, provide low-cost abstractions for creating more expressive, composable code. 
\par
Template metaprogramming, is a Turing-complete technique for describing compile-time computation in C\lstinline{++}. This technique will be of particular importance in the construction of the type parsing library. The boost spirit X3 library will be used for generating the recursive descent parsers. Boost spirit X3 constructs an approximation to the extended Backus-Naur form using C\lstinline{++} template metaprogramming.
\par
The proposed software tool will be a desktop GUI application, developed using the Electron framework. Electron will facilitate the use of powerful web technologies, such as React and D3, for building interactive graphical interfaces. Accordingly, the programming language Typescript will be used for developing the graphical interface. Typescript extends the Javascript language with optional static typing, which includes type-inference.
\par
The React library will be used in the development of user interfaces. React provides an approach to building interfaces by means of composition of smaller components. In conjunction with React, the D3 library will be used for creating the Petri net diagrams. D3 supplies a collection of functions which facilitate a streamlined approach to generating data visualisations. In the context of the proposed application, a particular advantage of D3 is the capacity to dynamically generate and animate scalable vector graphics (SVG). Importantly, D3 does not provide pre-made graphical components, but is a library intended for organising data visualisation.
\par
Interoperability between C\lstinline{++} and Typescript will be possible by means of creating node native modules from the C\lstinline{++} code. The meta-build tool CMake, will used for describing the build process in which node native modules are created. CMake offers the ability to output project files and build scripts for a range of different integrated development environments and build tools. The Node.js build tool will be used for building the entire Electron application. GoogleTest will be used in the unit testing of C\lstinline{++} code.

\end{document}